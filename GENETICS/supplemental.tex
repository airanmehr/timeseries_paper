%%%%%%%%%%%%%%%%%%%%%%%%%%%%%%%%%%%%%%%%%%%%%
%%%%%%%%%%%%%%%%%%% Supplemental Material%%%%
%%%%%%%%%%%%%%%%%%%%%%%%%%%%%%%%%%%%%%%%%%%%%
%\clearpage
%\newpage
\section{Choosing Window Size}
In genome-wide scans for detecting selection, we apply the \comale\
statistic on sliding windows of length $L$bp. The single locus
statistic values within the window are averaged to get the composite
statistic. While the statistic is robust to variation in window-size,
choosing a very large window where LD has decayed will weaken the
composite signal, and choosing a small window will decrease the power
of composite likelihoods. Here, we use a systematic calculation to
choose $L$ as the distance where the LD between the favored mutation
and a site $L/2$bp away remains strong.
\label{sec:winSize}
\ignore{
The optimum
choice of window size depends on the linkage disequilibrium (LD)
between the favored allele and its nearby variants.  In an E\&R
selection experiment, dynamic of the exact LD between the favored
allele to other variants depends on a number of parameters including
recombination rate ($r$), genomic distance ($l$), LD at the onset of
selection ($\rho_0$), initial frequency of the favored allele
($\nu_0$), strength of selection ($s$) and span of experiment
($\tau$). Moreover, observed LD, additionally depend on the sample
size $n$ and sequencing coverage. For simplicity we exclude sampling
noise due to finite sampling for sequencing and finite sequencing
coverage in our analysis.
}


Consider a segregating site $l$ bp away from the favored allele in a
selective sweep.  Let $\rho_\tau$ be the LD between the favored allele
and the site, $\tau$ generations after the onset of selection. Then,
we have (see Eqs. 30-31 in \cite{stephan2006hitchhiking}): 
\beq
\rho_\tau= \alpha_\tau\beta_\tau \rho_0=e^{-r\tau l}
\left(\frac{K^{(\tau)}}{K^{(0)}}\right)\rho_0\label{eq:ldt}, 
\eeq
where $K^{(\tau)}=2\nu_\tau(1-\nu_\tau)$ is the heterozygosity at the
selected site, $r$ is the recombination rate (crossovers/bp/gen). The
`decay factor', $\alpha_\tau=e^{-r\tau l}$, and `growth factor',
$\beta_\tau$, are due to recombination and selection, respectively.
Under regular parameter settings, linkage to the favored allele is
expected to increase after onset of selection and then 
decreases due to crossover events
(See~\ref{fig:winSize}-A).  While $\rho_0$ is
unknown in pool-seq E\&R experiments, we compute the value of $l$ so
that
\beq
\alpha_\tau \beta_\tau=1.\label{eq:eq}
\eeq
In E\&R scenarios, we let $\tau$ be the time of the last sampling. For
given $s$, we aim to compute the smallest window size $L$ over all
possible starting frequencies. Specifically,

\beq L=2\min_{\nu_0} \left\{
  \frac{1}{r\tau}\log\left(\frac{\hat{\nu}_\tau(1-\hat{\nu}_\tau)}{\nu_0(1-\nu_0)}\right)\right\},
  \label{eq:winSize}
\eeq 
where the term $\hat{\nu}_\tau$ depends on initial frequency
$\nu_0$ and selection strength $s$ (Eq.~\ref{eq:transition}).


We used \dmel dataset parameters, $N=250,r=2\times10^{-8}$ and
$\tau=59$ to compute the optimal window size for different values of
$Ns$, ranging from weak selection to strong selection:
$Ns\in\{20,100,200,500\}$, or $s\in\{0.08,0.4,0.8,2\}$.  We set
$L=30$Kbp (See~\ref{fig:winSize}-B) to provide good resolution for
detecting weak selection.








