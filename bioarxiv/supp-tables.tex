
\begin{table}[H]
	\centering
		\caption{\bf Average of power for detecting selection.}
	\begin{tabular}{ccc}
		Hard Sweep & &Soft Sweep\\ \\  
		\centering \begin{tabular}{c|c|c}
$\lambda$	&Method	&Avg Power\\\hline
100	&$\mathcal{H}$	&30\\
$\infty$	&$\mathcal{H}$	&29\\
30	&$\mathcal{H}$	&23\\
100	&CMH	&19\\
30	&CMH	&10\\
$\infty$	&GP	&7\\
100	&GP	&7\\
30	&GP	&7\\
$\infty$	&FIT	&5\\
100	&FIT	&4\\
30	&FIT	&3\\
\end{tabular}

		&&\centering \begin{tabular}{c|c|c}
$\lambda$	&Method	&Avg Power\\\hline
100	&$\mathcal{H}$	&64\\
$\infty$	&$\mathcal{H}$	&63\\
$\infty$	&GP	&60\\
100	&GP	&60\\
100	&CMH	&59\\
30	&$\mathcal{H}$	&58\\
30	&GP	&56\\
30	&CMH	&44\\
$\infty$	&FIT	&36\\
100	&FIT	&21\\
30	&FIT	&7\\
\end{tabular}

	\end{tabular}
	\caption*{		Average power is computed  for 8000 simulations 
	with 
		$s\in\{0.025,0.05,0.075,0.1\}$. Frequency Increment 
		Test (FIT), Gaussian Process (GP), \comale\ ($\Hc$ statistic) and 
		Cochran Mantel Haenszel (CMH) are compared for different initial 
		carrier frequency $\nu_0$. For all sequencing coverages, \comale\ 
		outperform other methods. When coverage is not high 
		($\lambda\in\{30,100\}$) and initial frequency is low (hard sweep), 
		\comale\ significantly perform better than others.}
	\label{tab:power}
\end{table}

\ignore{
\begin{table}[H]
	\centering
		\caption{\bf Average running time per variant in seconds for 
		different 
			methods.}
	\begin{tabular}{c}
		\centering \begin{tabular}{c|c}
Method	&Avg. Time per Variant\\\hline
CMH	&0.001\\
$\mathcal{M}$	&0.006\\
FIT	&0.006\\
$\mathcal{H}$	&0.042\\
GP(1)	&2.551\\
GP(3)	&19.177\\
GP(5)	&50.291\\
GP(7)	&95.602\\
GP(10)	&202.017\\
\end{tabular}

	\end{tabular}
\label{tab:times}
\end{table}
}

\begin{table}[h]
	\centering
	\caption{\bf Mean and standard deviation of the distribution of 
	bias 
		($s-\hat{s}$) of 8000 simulations with coverage 
		$\lambda=100\times$ and 
		$s\in\{0.025,0.05,0.075,0.1\}$.}
	\begin{tabular}{c}
		\centering \begin{tabular}{c|c|c|c|c}
index	&GP	&GP	&HMM	&HMM\\\hline
	&0.005	&0.1	&0.005	&0.1\\
	&	&	&	&\\
count	&2000.0	&2000.0	&2000.0	&2000.0\\
mean	&0.041	&0.0	&-0.003	&0.0\\
std	&0.036	&0.013	&0.028	&0.013\\
min	&-0.056	&-0.042	&-0.095	&-0.046\\
25\%	&0.018	&-0.009	&-0.02	&-0.009\\
50\%	&0.039	&-0.0	&-0.005	&-0.0\\
75\%	&0.062	&0.009	&0.015	&0.009\\
max	&0.25	&0.05	&0.1	&0.056\\
\end{tabular}

	\end{tabular}
	\label{tab:biasdist}
\end{table}
\begin{table}
		\caption{\bf Overlapping genes with the 174 candidate variants.}
	\label{tab:genes}
	\centering \begin{tabular}{c|c|c|c|p{3in}}
Interval	&Position	&FBgn	&Gene Name	&GO Function\\\hline
\multirow{8}{*}{I1}	&	\multirow{8}{*}{X:1.567-1.824M} &FBgn0023531	
&CG32809	& NA\\
& &FBgn0023130	&a6	& 
embryonic development via the syncytial blastoderm\\
	&	&FBgn0025378	&CG3795	& serine-type endopeptidase activity\\
	&	&FBgn0025391	&Scgdelta	& heart contraction, mesoderm development\\
	&	&FBgn0261548	&CG42666	&NA\\
	&	&FBgn0026086	&Adar	& RNA editing\\
	&	&FBgn0026090	&CG14812	& negative regulation of cysteine-type 
	endopeptidase activity involved in apoptotic process\\
	&	&FBgn0023522	&CG11596	&NA\\
	\hline
\multirow{3}{*}{I2}	&	\multirow{3}{*}{X:7.175-7.241M}	&FBgn0029941	&CG1677	
&\\
	&	&FBgn0029944	&Dok	& sress activated protein kinase signaling\\
	&	&FBgn0029946	&CG15034	&NA\\
\hline
\multirow{8}{*}{I3}	&	\multirow{8}{*}{2L:16.878-16.993M}	&FBgn0052832	
&CG32832	& mitochondrial pyruvate transport \\
	&	&FBgn0032618	&CG31743	& sulfotransferase  activity\\
	&	&FBgn0085342	&CG34313	&NA\\
	&	&FBgn0040985	&CG6115	&NA\\
	&	&FBgn0261671	&tweek	&  synaptic vesicle endocytosis\\
	&	&FBgn0026150	&ApepP	& metalloaminopeptidase activity\\
	&	&FBgn0262355	&CR43053	&NA\\
	&	&FBgn0053179	&beat-IIIb	&NA\\
	\hline
\multirow{7}{*}{I4}	&	\multirow{7}{*}{2R:2.725-2.810M}	&FBgn0040674	
&CG9445	&NA\\
	&	&FBgn0265935	&coro	& adult somatic muscle development\\
	&	&FBgn0033110	&CG9447	&NA\\
	&	&FBgn0033113	&Spn42Dc	& Inhibitory Serpins\\
	&	&FBgn0028988	&Spn42Dd	& Inhibitory Serpins\\
	&	&FBgn0033115	&Spn42De	& Inhibitory Serpins \\
	&	&FBgn0050158	&CG30158	& small GTPase mediated signal 
	transduction\\
	\hline
\multirow{6}{*}{I5}	&	\multirow{6}{*}{3L:14.362-14.514M}	&FBgn0036421	
&CG13481	& ubiquitin-protein transferase activity\\
	&	&FBgn0262580	&CG43120	& NA \\
	&	&FBgn0036422	&CG3868	& NA\\
	&	&FBgn0087007	&bbg	& PDZ domain\\
	&	&FBgn0036426	&CG9592	& NA \\
	&	&FBgn0036427	&CG4613	& serine-type endopeptidase activity \\
\end{tabular}

\end{table}



%\begin{table}
%		\caption{\bf Overlapping genes with the 174 candidate variants.}
%		\label{tab:genes}
%	\centering \begin{tabular}{c|c|c|c|c|c | p{4in}}
index	&FBgn	&CHROM	&start	&end	&name & Function\\\hline
1	&FBgn0052832	&2L	&16878326	&16879290	&CG32832& mitochondrial 
pyruvate transport \\
2	&FBgn0032618	&2L	&16879517	&16886319	&CG31743 & sulfotransferase 
activity\\
3	&FBgn0085342	&2L	&16879517	&16886319	&CG34313&\\
4	&FBgn0040985	&2L	&16887109	&16887966	&CG6115 &\\
5	&FBgn0261671	&2L	&16888490	&16917052	&tweek&  synaptic vesicle 
endocytosis\\
6	&FBgn0026150	&2L	&16908229	&16910418	&ApepP& metalloaminopeptidase 
activity\\
7	&FBgn0262355	&2L	&16944723	&16945374	&CR43053&\\
8	&FBgn0053179	&2L	&16973091	&16993984	&beat-IIIb&\\
9	&FBgn0040674	&2R	&2725579	&2726560	&CG9445\\
10	&FBgn0265935	&2R	&2749506	&2760223	&coro & adult somatic muscle 
development\\
11	&FBgn0033110	&2R	&2760501	&2763324	&CG9447\\
12	&FBgn0033113	&2R	&2768500	&2770912	&Spn42Dc& Inhibitory Serpins\\
13	&FBgn0028988	&2R	&2770785	&2772378	&Spn42Dd& Inhibitory Serpins\\
14	&FBgn0033115	&2R	&2773057	&2775767	&Spn42De& Inhibitory Serpins \\
15	&FBgn0050158	&2R	&2779265	&2810118	&CG30158 & small GTPase 
mediated signal transduction\\
16	&FBgn0036421	&3L	&14362025	&14362807	&CG13481 & ubiquitin-protein 
transferase activity\\
17	&FBgn0262580	&3L	&14375013	&14376399	&CG43120 & \\
18	&FBgn0036422	&3L	&14393869	&14395825	&CG3868 & \\
19	&FBgn0087007	&3L	&14405928	&14529376	&bbg & PDZ domain\\
20	&FBgn0036426	&3L	&14510925	&14511575	&CG9592 & \\
21	&FBgn0036427	&3L	&14512860	&14514790	&CG4613 & serine-type 
endopeptidase activity \\
22	&FBgn0023531	&X	&1567143	&1586801	&CG32809 & \\
23	&FBgn0023130	&X	&1587648	&1589922	&a6 & embryonic development via 
the syncytial blastoderm\\
24	&FBgn0025378	&X	&1602839	&1604215	&CG3795 & serine-type 
endopeptidase activity\\
25	&FBgn0025391	&X	&1629978	&1648098	&Scgdelta & heart contraction, 
mesoderm development\\
26	&FBgn0261548	&X	&1667752	&1747700	&CG42666\\
27	&FBgn0026086	&X	&1667758	&1682098	&Adar & RNA editing\\
28	&FBgn0026090	&X	&1751922	&1753004	&CG14812 & negative regulation 
of cysteine-type endopeptidase activity involved in apoptotic process\\
29	&FBgn0023522	&X	&1821716	&1824550	&CG11596&\\
30	&FBgn0029941	&X	&7175122	&7299830	&CG1677\\
31	&FBgn0029944	&X	&7218247	&7222839	&Dok & sress activated protein 
kinase signaling\\
32	&FBgn0029946	&X	&7240292	&7241539	&CG15034\\
\end{tabular}

%\end{table}

\ignore{
\begin{table}
		\caption{aa}
		\label{tab:go-reg}
	\centering \begin{tabular}{c|c|c|c|c|p{2in}}
GO ID	&-log($p$-value)	&Hits	&Num of Genes	&Total Genes	&GO Term\\\hline
GO:0019233	&1.6237e-06	&4	&435	&552	&sensory perception of pain\\
GO:0005615	&1.6237e-06	&5	&306	&405	&extracellular space\\
GO:0004867	&1.6237e-06	&4	&48	&83	&serine-type endopeptidase inhibitor activity\\
GO:0045861	&1.6237e-06	&4	&16	&21	&negative regulation of proteolysis\\
\end{tabular}

\end{table}
}