\begin{abstract}
  The advent of next generation sequencing technologies has made
  whole-genome and whole-population sampling possible, even for
  eukaryotes with large genomes. With this development, experimental
  evolution studies can be designed to observe molecular evolution
  ``in-action'' via Evolve-and-Resequence (E\&R) experiments. Among
  other applications, E\&R studies can be used to locate the genes and
  variants responsible for genetic adaptation. Most of existing literature on
  time-series data analysis often assumes large population size,
  accurate allele frequency estimates, or wide time spans. These
  assumptions do not hold in many E\&R studies.
	
  In this article, we propose a method--Composition of Likelihoods for
  Evolve-And-Resequence experiments (\comale)--to identify signatures of 
  selection
in small population E\&R experiments. \comale\ takes 
whole-genome sequence of
  pool of individuals (pool-seq) as input, and properly addresses
  heterogeneous ascertainment bias resulting from uneven coverage.
  \comale\ also provides unbiased estimates of model parameters,
  including population size, selection strength and dominance,
  while being computationally efficient.  Extensive simulations show
  that \comale\ achieves higher power in detecting and localizing
  selection over a wide range of parameters, and is robust to
  variation of coverage.  We applied \comale\ statistic to multiple
  E\&R experiments, including, \datadm\ and a study of outcrossing
  yeast populations, and identified multiple regions under selection
  with genome-wide significance.
\end{abstract}