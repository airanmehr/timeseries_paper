\section{Discussion}
We developed a computational tool, \comale, that can detect regions
and variants under selection E\&R experiments. Using extensive simulations, 
we show that \comale\
outperforms existing methods in detecting selection, locating the
favored allele, and estimating model parameters.  Also, while
being computationally efficient, \comale\ provide means for estimating
populations size and hypothesis testing.

Many factors such as small population size, finite coverage, linkage
disequilibrium, finite sampling for sequencing, duration of the
experiment and the small number of replicates can limit the power of
tools for analyzing E\&R.  Here, by an discrete modeling, \comale\
estimates population size, and provides unbiased estimates of
$s,h$. It adjusts for heterogeneous coverage of pool-seq data, and
exploits presence of linkage within a region to compute composite
likelihood ratio statistic.


It should be noted that, even though we described \comale\ for small
fixed-size populations, the statistic can be adjusted for other
scenarios, including changing population sizes when the demography is
known. For large populations, transitions can be computed on sparse
data structures, as for large $N$ the transition matrices become
increasingly sparse. Alternatively, frequencies can be binned to
reduce dimensionality.


The comparison of hard and soft sweep scenarios showed that initial
frequency of the favored allele can have an nontrivial effect on the
statistical power for identifying selection. Interestingly, while it
is easier to detect a region undergoing strong selection, it is harder
to locate the favored allele in that region.


There are many directions to improve the analyses presented here.  In
particular, we plan to focus our attention on other organisms with
more complex life cycles, experiments with variable population size
and longer sampling-time-spans. As evolve and resequencing experiments
continue to grow, deeper insights into adaptation will go hand in hand
with improved computational analysis.



