\begin{table}[h]
	\centering
	\begin{tabular}{ccc}
		Hard Sweep & &Soft Sweep\\ \\  
		\centering \begin{tabular}{c|c|c}
$\lambda$	&Method	&Avg Power\\\hline
100	&$\mathcal{H}$	&24\\
$\infty$	&$\mathcal{H}$	&23\\
30	&$\mathcal{H}$	&18\\
100	&CMH	&15\\
30	&CMH	&8\\
$\infty$	&GP	&6\\
100	&GP	&6\\
30	&GP	&6\\
$\infty$	&FIT	&5\\
100	&FIT	&3\\
30	&FIT	&3\\
\end{tabular}

		&&\centering \begin{tabular}{c|c|c}
$\lambda$	&Method	&Avg Power\\\hline
300	&CMH	&71\\
300	&\sc{Clear}	&69\\
300	&GP	&63\\
300	&FIT	&8\\
100	&\sc{Clear}	&66\\
100	&CMH	&63\\
100	&GP	&61\\
100	&FIT	&1\\
30	&GP	&59\\
30	&\sc{Clear}	&56\\
30	&CMH	&45\\
30	&FIT	&2\\
\end{tabular}

	\end{tabular}
        \caption{Average of power for detecting selection in a 50Kbp region, 
        when power is computer for 1000 simulations with 
        $s\in\{0.025,0.05,0.075,0.1\}$. Frequency Increment 
        Test (FIT), Gaussian Process (GP), \comale\ ($\Hc$) and 
        Cochran–Mantel–Haenszel (CMH) are compared for different initial 
        carrier frequency $\nu_0$. For all sequencing coverages, \comale\ 
        outperform other methods. When coverage is not high 
        ($\lambda\in\{30,100\}$) and initial frequency is low (hard sweep), 
        \comale\ significantly perform better than others.}
\label{tab:power}
\end{table}


\begin{table}[h]
	\centering
	\begin{tabular}{c}
		\centering \begin{tabular}{c|c}
Method	&Avg. Time per Locus\\\hline
CMH	&0.0\\
$H$	&0.003\\
FIT	&0.006\\
GP(1)	&2.551\\
GP(3)	&19.177\\
GP(5)	&50.291\\
GP(7)	&95.602\\
GP(10)	&202.017\\
\end{tabular}

	\end{tabular}
	\caption{Average running time per variant in seconds for different 
	methods.}\label{tab:times}
\end{table}

\begin{table}[h]
	\centering
	\begin{tabular}{c}
		\centering \begin{tabular}{c|c|c|c|c}
index	&GP	&GP	&HMM	&HMM\\\hline
	&0.005	&0.1	&0.005	&0.1\\
	&	&	&	&\\
count	&2000.0	&2000.0	&2000.0	&2000.0\\
mean	&0.055	&0.002	&0.041	&0.008\\
std	&0.04	&0.014	&0.026	&0.013\\
min	&-0.053	&-0.038	&-0.045	&-0.039\\
25\%	&0.025	&-0.008	&0.025	&-0.001\\
50\%	&0.05	&0.002	&0.04	&0.008\\
75\%	&0.075	&0.012	&0.055	&0.017\\
max	&0.25	&0.051	&0.1	&0.058\\
\end{tabular}

	\end{tabular}
	\caption{Mean and standard deviation of the distribution of bias 
	($s-\hat{s}$) of 2000 simulations with coverage $\lambda=100\times$ and 
	$s\in\{0.025,0.05,0.075,0.1\}$.}\label{tab:biasdist}
\end{table}


\begin{table}[h]
	\centering
	\begin{tabular}{c}
		\centering \begin{tabular}{l|r}
Statistic	&Value\\\hline
Num. of Vatiants	&1,608,032\\
Num. of Candidate Intervals	&89\\
Total Num. of Genes	&17,293\\
Num. of Variant Genes	&12,834\\
Num. of Genes within Candidate Intervals	&968\\
Total Num. of GO	&6,983\\
Num. of GO with 3 or More Genes	&3,447\\
Num. of Candidate Variants for Gowinda	&2,886\\
\end{tabular}

	\end{tabular}
	\caption{.}\label{tab:stats}
\end{table}