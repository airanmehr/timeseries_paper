\begin{table}[h]
	\centering
	\begin{tabular}{ccc}
		Hard Sweep & &Soft Sweep\\ \\  
		\centering \begin{tabular}{c|c|c}
$\lambda$	&Method	&Avg Power\\\hline
100	&$\mathcal{H}$	&30\\
$\infty$	&$\mathcal{H}$	&29\\
30	&$\mathcal{H}$	&23\\
100	&CMH	&19\\
30	&CMH	&10\\
$\infty$	&GP	&7\\
100	&GP	&7\\
30	&GP	&7\\
$\infty$	&FIT	&5\\
100	&FIT	&4\\
30	&FIT	&3\\
\end{tabular}

		&&\centering \begin{tabular}{c|c|c}
$\lambda$	&Method	&Avg Power\\\hline
100	&$\mathcal{H}$	&64\\
$\infty$	&$\mathcal{H}$	&63\\
$\infty$	&GP	&60\\
100	&GP	&60\\
100	&CMH	&59\\
30	&$\mathcal{H}$	&58\\
30	&GP	&56\\
30	&CMH	&44\\
$\infty$	&FIT	&36\\
100	&FIT	&21\\
30	&FIT	&7\\
\end{tabular}

	\end{tabular}
        \caption{Average of power for detecting selection in a 50Kbp region, 
        when power is computer for 1000 simulations with 
        $s\in\{0.025,0.05,0.075,0.1\}$. Frequency Increment 
        Test (FIT), Gaussian Process (GP), \comale\ ($\Hc$) and 
        Cochran–Mantel–Haenszel (CMH) are compared for different initial 
        carrier frequency $\nu_0$. For all sequencing coverages, \comale\ 
        outperform other methods. When coverage is not high 
        ($\lambda\in\{30,100\}$) and initial frequency is low (hard sweep), 
        \comale\ significantly perform better than others.}
\label{tab:power}
\end{table}


\begin{table}[h]
	\centering
	\begin{tabular}{c}
		\centering \begin{tabular}{c|c|c|c}
 	&CHROM	&start	&end\\\hline
1	&3L	&14330000	&14420000\\
2	&3L	&14490000	&14540000\\
3	&3R	&10640000	&10730000\\
4	&3R	&11770000	&11850000\\
5	&3R	&11980000	&12240000\\
6	&3R	&12250000	&12330000\\
7	&3R	&12360000	&12410000\\
8	&3R	&12780000	&12900000\\
9	&3R	&13040000	&13100000\\
10	&3R	&13160000	&13220000\\
11	&3R	&14380000	&14460000\\
12	&3R	&14470000	&14520000\\
13	&3R	&14540000	&14590000\\
14	&3R	&14700000	&14760000\\
15	&3R	&14920000	&14980000\\
16	&3R	&17240000	&17390000\\
17	&3R	&18300000	&18360000\\
18	&3R	&18630000	&18690000\\
19	&3R	&20050000	&20130000\\
20	&3R	&20230000	&20280000\\
21	&3R	&20400000	&20450000\\
\end{tabular}

	\end{tabular}
	\caption{Genomic coordinates (FlyBase assembly release 5.7) of the regions 
		identified by \comale\ as being under selection in the 
		\datadm.}\label{tab:intervals}
\end{table}

\begin{table}[h]
	\centering
	\begin{tabular}{c}
		\centering \begin{tabular}{c|c}
Method	&Avg. Time per Variant\\\hline
CMH	&0.001\\
$\mathcal{M}$	&0.006\\
FIT	&0.006\\
$\mathcal{H}$	&0.042\\
GP(1)	&2.551\\
GP(3)	&19.177\\
GP(5)	&50.291\\
GP(7)	&95.602\\
GP(10)	&202.017\\
\end{tabular}

	\end{tabular}
	\caption{Average running time per variant in seconds for different 
	methods.}\label{tab:times}
\end{table}

\begin{table}[h]
	\centering
	\begin{tabular}{c}
		\centering \begin{tabular}{c|c|c|c|c}
index	&GP	&GP	&HMM	&HMM\\\hline
	&0.005	&0.1	&0.005	&0.1\\
	&	&	&	&\\
count	&2000.0	&2000.0	&2000.0	&2000.0\\
mean	&0.041	&0.0	&-0.003	&0.0\\
std	&0.036	&0.013	&0.028	&0.013\\
min	&-0.056	&-0.042	&-0.095	&-0.046\\
25\%	&0.018	&-0.009	&-0.02	&-0.009\\
50\%	&0.039	&-0.0	&-0.005	&-0.0\\
75\%	&0.062	&0.009	&0.015	&0.009\\
max	&0.25	&0.05	&0.1	&0.056\\
\end{tabular}

	\end{tabular}
	\caption{Mean and standard deviation of the distribution of bias 
	($s-\hat{s}$) of 2000 simulations with coverage $\lambda=100\times$ and 
	$s\in\{0.025,0.05,0.075,0.1\}$.}\label{tab:biasdist}
\end{table}