\begin{table}[h]
	\centering
	\begin{tabular}{ccc}
		Hard Sweep & &Soft Sweep\\ \\  
		\centering \begin{tabular}{c|c|c}
$\lambda$	&Method	&Avg Power\\\hline
100	&$\mathcal{H}$	&24\\
$\infty$	&$\mathcal{H}$	&23\\
30	&$\mathcal{H}$	&18\\
100	&CMH	&15\\
30	&CMH	&8\\
$\infty$	&GP	&6\\
100	&GP	&6\\
30	&GP	&6\\
$\infty$	&FIT	&5\\
100	&FIT	&3\\
30	&FIT	&3\\
\end{tabular}

		&&\centering \begin{tabular}{c|c|c}
$\lambda$	&Method	&Avg Power\\\hline
300	&CMH	&71\\
300	&\sc{Clear}	&69\\
300	&GP	&63\\
300	&FIT	&8\\
100	&\sc{Clear}	&66\\
100	&CMH	&63\\
100	&GP	&61\\
100	&FIT	&1\\
30	&GP	&59\\
30	&\sc{Clear}	&56\\
30	&CMH	&45\\
30	&FIT	&2\\
\end{tabular}

	\end{tabular}
        \caption{Average of power for detecting selection in a 50Kbp region, 
        when power is computer for 1000 simulations with 
        $s\in\{0.025,0.05,0.075,0.1\}$. Frequency Increment 
        Test (FIT), Gaussian Process (GP), \comale\ ($\Hc$) and 
        Cochran–Mantel–Haenszel (CMH) are compared for different initial 
        carrier frequency $\nu_0$. For all sequencing coverages, \comale\ 
        outperform other methods. When coverage is not high 
        ($\lambda\in\{30,100\}$) and initial frequency is low (hard sweep), 
        \comale\ significantly perform better than others.}
\label{tab:power}
\end{table}


\begin{table}[h]
	\centering
	\begin{tabular}{c}
		\centering \begin{tabular}{c|c|c|c}
 	&CHROM	&start	&end\\\hline
1	&2R	&20970000	&21110000\\
2	&3L	&14200000	&14250000\\
3	&3L	&14320000	&14440000\\
4	&3L	&14450000	&14550000\\
5	&3L	&15660000	&15720000\\
6	&3R	&10640000	&10730000\\
7	&3R	&11780000	&11840000\\
8	&3R	&12010000	&12240000\\
9	&3R	&12250000	&12340000\\
10	&3R	&12820000	&12900000\\
11	&3R	&13040000	&13100000\\
12	&3R	&13170000	&13220000\\
13	&3R	&14390000	&14440000\\
14	&3R	&14470000	&14520000\\
15	&3R	&14530000	&14590000\\
16	&3R	&14920000	&14980000\\
17	&3R	&16740000	&16800000\\
18	&3R	&17250000	&17330000\\
19	&3R	&17340000	&17390000\\
20	&3R	&18630000	&18680000\\
21	&3R	&18880000	&18940000\\
22	&3R	&20050000	&20120000\\
23	&3R	&20230000	&20280000\\
24	&3R	&20400000	&20450000\\
25	&3R	&20540000	&20600000\\
\end{tabular}

	\end{tabular}
	\caption{Genomic coordinates (FlyBase assembly release 5.7) of the regions 
		identified by \comale\ as being under selection in the 
		\datadm.}\label{tab:intervals}
\end{table}

\begin{table}[h]
	\centering
	\begin{tabular}{c}
		\centering \begin{tabular}{c|c}
Method	&Avg. Time per Locus\\\hline
CMH	&0.0\\
$H$	&0.003\\
FIT	&0.006\\
GP(1)	&2.551\\
GP(3)	&19.177\\
GP(5)	&50.291\\
GP(7)	&95.602\\
GP(10)	&202.017\\
\end{tabular}

	\end{tabular}
	\caption{Average running time per locus in seconds for different 
	methods.}\label{tab:times}
\end{table}