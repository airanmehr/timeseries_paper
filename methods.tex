\section{Materials and Methods}
\label{sec:method}
\paragraph{Notation.} 
Consider a locus with starting derived allele frequency
$\nu_0$. Frequencies are sampled at $T+1$ distinct generations specified
by ${\cal T}= \{\tau_i: 0\le \tau_0<\tau_1,\ldots\le \tau_T\}$, and
denoted by $\bm{\nu}=\{\nu_0,\ldots,\nu_T\}$. Moreover, $R$ replicate
measurements are made, and we denote the $r$-th replicate frequency
data as $\bm{\nu}^{(r)}$.

\subsection{The \comale\  statistic}
To test if a genomic region is evolving under natural selection, we
consider a likelihood-based 
approach~\cite{vitti2013detecting,nielsen2005genomic,Terhorst2015Multi} to 
 (a) maximizes the likelihood of the time series data
w.r.t. selection and overdominance parameters $s,h$, for each polymorphism in 
the region; and, 
(b) computes
the log-odds ratio of the likelihood of selection model to the
likelihood of neutral evolution/drift model, for every polymorphism in 
the region. 
Then, (c) site likelihood ratios are combined to compute composite likelihood 
for a region, \comale\ statistic. 
Thus, in addition to classifying a window as being under selection or not, 
computing \comale\ statistic for a region also provides estimate of 
selection parameters as well as a ranking of variants within the testing 
region. 

\paragraph{Likelihood for Neutral Model.}
To model neutral evolution, it is natural to model the change in
frequency $\nu_t$ over time via Brownian
motion~\cite{feder2014Identifying} or Gaussian
process~\cite{Terhorst2015Multi}. Significant deviations from this
Null could be indicative of non-neutrality. However, in our
experiments, we found that the Brownian motion approximation is
inadequate for small populations, low starting frequencies and sparse sampling 
(in time)
that are typical in experimental evolution (see Results, and
Fig.~\ref{fig:markov}). In fact, other ``continuous" models such as
Gaussian process for dynamic allele frequencies, are also susceptible to this
issue.

Instead, by computing likelihood of data using a discrete-time
discrete-state-space Wright-Fisher Markov chain, we turn the problem
of small-population size into an advantage. Consider a neutrally
evolving diploid population with $N$ individuals. Define a
$2N\times2N$ transition matrix $P$, where $P^{(\tau)}[i,j]$ denotes
probability of change in allele frequency from $\frac{i}{2N}$ to
$\frac{j}{2N}$ in $\tau$ generations, solely due to genetic drift. $P$
is defined as follows~\cite{Ewens2012Mathematical}:
\begin{eqnarray}
  P^{(1)}[i,j] &=& \pr\left(\nu_{t+1}=\frac{j}{2N} \left|
      \nu_{t}=\frac{i}{2N}\right)={2N \choose j} \right.  \nu_{t}^j
  (1-\nu_{t})^{2N-j}, \;\;\label{eq:P1}\\
  P^{(\tau)} &=&   P^{(\tau-1)}P^{(1)} \label{eq:Pt}
\end{eqnarray}
Note that $P^{(\tau)}$ needs to be computed only once and can be reused for all 
the variants in the genome. Also, precomputing and storing $P^{(\tau)}$ is 
tractable and
numerically stable for controlled experimental evolution experiments
with $N\le5000$. 

\paragraph{Likelihood for Selection Model.}
Assume that the site is evolving under selection constraints $s\in
\Rbb$, $h\in \Rbb_+$, where $s$ and $h$ denote selection strength and 
overdominance parameters ,
respectively. By definition, the relative fitness values of genotypes
0$|$0, 0$|$1 and 1$|$1 are given by $w_{00}=1$, $w_{01}=1+hs$ and
$w_{11}=1+s$. Recall that $\nu_t$ denotes the frequency of the site at
time $\tau_t\in {\cal T}$. Then, $\nusp$, the frequency at time
$\tau_{t}+1$ (one generation ahead) can be estimated using: \beq 
\hat{\nu}_{t^+} =
\mathbb{E}[\nusp|s,h,\nu_t]&=\frac{w_{11}\nu_t^2 +
  w_{01}\nu_t(1-\nu_t)}{w_{11}\nu^2_t + 2w_{01}\nu_t(1-\nu_t) +
  w_{00}(1-\nu_t)^2}\\
&=\nu_t+\frac{s(h+(1-2h)\nu_t)\nu_t(1-\nu_t)}{1+s\nu_t(2h+(1-2h)\nu_t))}.
  \label{eq:transition}
\eeq
For finite populations, let $Q^{(\tau)}_{s,h}[i,j]$ denote the
probability of transition from $\frac{i}{2N}$ to $\frac{j}{2N}$ in
$\tau$ generations. We model $Q$ as follows
(See~\cite{Ewens2012Mathematical}, Pg.~24, Eqn.~$1.58$-$1.59$):
\begin{eqnarray}
  Q^{(1)}_{s,h}[i,j] &=& \pr\left(\nusp=\frac{j}{2N} \left\lvert
      \nu_{t}=\frac{i}{2N};s,h \right .\right)={2N \choose j}
  \hat{\nu}_{t^+}^{j} (1-\hat{\nu}_{t^+})^{2N-j}\label{eq:Q1}\\
  Q^{(\tau)}_{s,h} &=& Q^{(\tau-1)}_{s,h}Q^{(1)}_{s,h}\label{eq:Qt}
  \label{eq:mkvs}   
\end{eqnarray}
For $s=0$, Eq.~\ref{eq:Q1} and~\ref{eq:Qt} are identical to
Eq.~\ref{eq:P1} and~\ref{eq:Pt}, respectively.  The likelihood of
observing the trajectory $\bm{\nu}$ is computed using:
\begin{equation}
  \Lc_M(s,h|\bm{\nu}) = \pr(\bm{\nu};s,h)=
  \prod_{t=1}^{T} \pr(\nu_{t}|\nu_{t-1};s,h) = \prod_{t=1}^{T} 
  Q^{(\delta_t)}_{s,h}[\hat{i},\hat{j}],
%  \label{eq:mkvlik}  
\end{equation}
where, $(\hat{i},\hat{j})=( 2N\nu_{t-1},
2N\nu_{t})$\AI{if population size remain 
constant, $2N\nu$ is always an integer. right?}, and 
$\delta_t=\tau_{t}-\tau_{t-1}$. 
Combining the
likelihood over independent replicate samples $\bm{\nu}^{(r)}$, we
get:
\begin{equation}
  \Lc_M(s,h|\{\bm{\nu}^{(r)}\}) = \prod_r   \Lc_M(s,h|\bm{\nu}^{(r)}).
  \label{eq:mkvlik}
\end{equation}
Let $\hat{s},\hat{h}$ denote the parameters that maximize the
likelihood. The simplest form of the Markov likelihood ratio test statistic  
for each variant (\comale-M) is given by
\begin{eqnarray}
M=\sgn (\hat{s}). \log 
\left(\frac{\Lc_M(\hat{s},\hat{h}|\{\bm{\nu}^{(r)}\})}{\Lc_M(0,0|\{\bm{\nu}^{(r)}\})}\right).
\label{eq:mcts}
\end{eqnarray}
%where the sign $\sgn(\hat{s})$ helps focus on positive selection using a 
%one-sided test.  




\paragraph{Accounting for Heterogeneous Ascertainment Bias.}
In the discussion so far, we assumed that the exact allele frequencies
are supplied. However, in most cases, allele frequencies are estimated
from genotype data with finite samples or pooled-sequencing data with
finite depth of coverage (Suppl.~Fig.~\ref{fig:stateConditional}).
Moreover, the depth at a site varies for different replicates, and
different time samples (Suppl.~Fig.~\ref{fig:depthHetero}). To account
for this heterogeneity, we extend the Markov chain likelihood in
Eq.~\ref{eq:mkvlik} to a Hidden Markov Model (HMM) for pool-seq data.

Consider a variant position being sampled at time point $\tau_t\in
{\cal T}$. We denote the pooled-seq data for that variant as $x_t =
\langle c_t,d_t \rangle$ where $d_t, c_t$ represent the read depth,
and the count of the derived allele, respectively, at time
$\tau_t$. The time-series data is represented by the sequence
$\bm{x}=\{x_0,x_1,x_2,\ldots,x_T\}$. We model the dynamic pool-seq data using a 
an HMM with $2N+1$ states which state $i$ $(0\le i\le 2N)$ corresponds to 
allele frequency $i/2N$. 
Also, we note that HMM is stationary model in that transition and emission
distributions do not change over time and defining these distributions 
completely specifies the HMM model. Eqs.~\ref{eq:P1},~\ref{eq:Pt} define 
transition probabilities and the probability that state $i$ emits $
x=\langle d, c\rangle $ is given by
\begin{equation*}
{\bf e}_{i}(x) = {d \choose c} \left(\frac{i}{2N} \right)^c\left (1- 
\frac{i}{2N} \right)^{d-c}.
\end{equation*}
For $1\le t\le T$, let $\alpha_{t,i}$ denote the probability of
emitting the $x_1,x_2,\ldots,x_t$ and ending in state $i$ at
$\tau_t$. Then, $\alpha_{t,i}$ can be computed using the
forward-procedure~\cite{durbin1998biological}:
\begin{equation}
  \alpha_{t,i} = \left( \sum_{1\le j\le 2N} 
  \alpha_{t-1,j}\;Q^{(\delta_t)}_{s,h}[j,i] \right) {\bf e}_{i}(x_t)\;\; .
  \label{eq:hmm}
\end{equation}
where $\delta_t=\tau_t-\tau_{t-1}$. The joint likelihood of the
observed data from $R$ independent observations is given by
\begin{equation}
  \Lc_{H}(s,h|\{\bm{x}^{(r)}\})=\prod_{r=1}^R\Lc_{H}(s,h|\bm{x}^{(r)}) = 
  \prod_{r=1}^R \sum_i\alpha_{T,i}^{(r)}\;\;.
  \label{eq:hmmlik}
\end{equation}
Similar to Eq.~\ref{eq:mcts}, let $\hat{s},\hat{h}$ denote the parameters that
maximize likelihood. The likelihood ratio statistic for
each variant of pool-seq data(\comale-H) is given by
\begin{eqnarray}
H &=& \sgn (\hat{s}). \log 
\left(\frac{\Lc_H(\hat{s},\hat{h}|\{\bm{x}^{(r)}\})}{\Lc_H(0,0|\{\bm{x}^{(r)}\})}\right).
\label{eq:hmmml}
\end{eqnarray}

\paragraph{Composite Likelihood Ratio.}
In general, the favored allele can be in linkage disequilibrium with some of
its surrounding variation. The linked-loci ``hitchhike" and share similar 
dynamics with the favored allele.
Some models such as Gaussian process~\cite{Terhorst2015Multi} take into account 
of these non-random associations to compute likelihood of linked-loci by 
modeling linkage and recombination explicitly. However, these approaches are 
computationally 
expensive and infeasible for pool-seq data as computing linkage 
requires phased haplotype data of the population.
Instead, a Composite Likelihood Ratio 
(CLR)~\cite{nielsen2005genomic,williamson2007localizing} score is computed from 
the individual scores of all variants in the region selection.


Consider a genomic region $L$ to be a collection of segregating sites with 
little or no recombination between sites and the favored allele, i.e., starting 
frequency of the favored allele is not high and the window around it is small 
enough. 
Let $M_\ell$ (respectively, $H_\ell$) denote the likelihood ratio
score based on Markov chain (respectively, HMM) for each site $\ell$
in $L$, the the classical CLR is computed by averaging scores of all the 
variants withing the testing region.
However, the levels of LD between favored allele and its surrounding variation 
depend on initial frequency $\nu_0$, strength of selection $s$ and time since 
the onset of selection (see Appendix~\ref{sec:ld} for more details).
Hence, we parameterize CLR to discard those polymorphisms that are in low LD 
with the favored allele, from computation of CLR.
In particular, we can choose to only include sites whose likelihood ratio
score is above a certain threshold. For percentile cut-off $\pi$, let
$L_{\pi}\subseteq L$ denote the set of sites whose likelihood ratio
scores had percentile $\pi$ or better. For all $\pi$, the modified CLR
statistic for Markov chain and HMM is computed using:
\beq
  {\cal M}_{\pi} = \frac{1}{|L_{\pi}|}\sum_{\ell \in L_{\pi}}M_\ell,
  &\hspace{0.5in}{\cal H}_{\pi} &=& \frac{1}{|L_{\pi}|}\sum_{\ell \in L_{\pi}} 
  H_\ell.
  \label{eq:pihmm}
\eeq
In summary, we use \comale-M to denote Markov Chain based likelihood
scores, and \comale-H to denote HMM based likelihood scores. We use
$M$, Eq.~\ref{eq:mcts} (respectively, $H$, Eq.~\ref{eq:hmmml}) to
denote the \comale\ scores for \emph{each variant}, and $\Mc_{\pi}$
(respectively, $\Hc_{\pi}$) to denote the \comale\ scores for a
\emph{window}. Note that ${\cal M}_{100},{\cal H}_{100}$ (respectively,
${\cal M}_0,{\cal H}_0$) correspond to the CLR after choosing the best
(respectively, all) site(s) in the region. 

\paragraph{Site-localization.} 
In general, localizing the adaptive allele in pool-seq data is 
difficult~\cite{tobler2014massive}, due to extensive span of hitchhikers in an 
ongoing sweep (see Appendix~\ref{sec:ld} for more detail). As a result, it 
is more reasonable to compute a set of 
``candidate" SNPs using a FDR cutoff found on the distribution of negative 
controls.  

\paragraph{Estimating parameters.}
\label{sec:regression}
Depending on data (read count or allele frequency) the optimal value
of the parameters can be found by 
\beqn
\hat{s},\hat{h}&=&\underset{s,h}{\arg\max} \sum_r^R \log
\left(\Lc_{\cal M}(s,h|\bm{\nu}^{(r)}\right),\;\; \mbox{ or, }\label{eq:mcmle}\\
\hat{s},\hat{h}&=&\underset{s,h}{\arg\max} \sum_r^R \log
\left(\Lc_{\cal H}(s,h|\bm{x}^{(r)}\right).\label{eq:hmmmle}
\eeqn
where likelihoods are defined in Eq.~\ref{eq:mkvlik} and
Eq.~\ref{eq:hmmlik}, respectively.  The parameters in
Eqs.~\ref{eq:mcmle},~\ref{eq:hmmmle} are optimized using grid search. By 
broadcasting and vectorizing the grid search operations across all variants, 
the genome scan on millions of polymorphisms can be done in 
significantly smaller time than iterating a numerical optimization routine for 
each variant(see Results and Fig.~\ref{fig:runTime}).

\paragraph{Precomputing Transition Matrices.}
\comale\ requires precomputation of matrices $Q^{(\tau)}_{s,h}$ for
the entire range of $s,h$ values. Precomputation of $909$ transition
matrices for $s\in\{-0.5,-0.49,\ldots,0.5 \}$ and $h\in
\{0,0.25,\ldots,2\}$ took less that 15 minutes ($\approx$ 1 second
per matrix) on a desktop computer with a Core i7 CPU and 16GB of RAM.


\subsection{Extending Site Frequency Spectrum  based tests for  time series 
data}\label{sec:extending-sfs}
\label{sec:sfs-ts}
The site frequency spectrum (SFS) is a mainstay of tests of neutrality
and selection, and can be computed using pool-seq data (does not
need haplotypes). Following Fu, 1995~\cite{fu1995statistical}, any
linear combination of the site frequencies is an estimate of
$\theta$. However, under non-neutral conditions, different linear
combinations behave differently. Therefore, many popular tests of
neutrality either compute differences of two estimates of $\theta$, or
perform cross-population tests comparing the $\theta$ estimates in two
different
populations~\cite{achaz2009frequency,ronen2013learning,sabeti2007genome}. 

We asked if SFS-based tests could be adapted for time-series data. A
simple approach is to use cross-population SFS tests on the
populations at time $0$ (before onset of selection), and at time
sample $\tau_t$, for each $t$. However, these tests are not
independent. Evans \emph{et al.}~\cite{evans2007non} developed
diffusion equations for evolution of SFS in time series, but they are
difficult to solve. Instead, we derive a formula for computing $D_t$,
the dynamic of Tajima's $D$ at generation $t$. Specifically, for
initial value $D_0$, initial carrier frequency $\nu_0$ and 
selection coefficient $s$:
\begin{equation}
  D_t=D_0-\log(1-\nu_t) \frac{W_0}{\log(2N)} -\nu_t^2 \Pi_0\;,
  \label{eq:tdt}    
\end{equation}
where $W_0$ and $\Pi_0$ are Watterson's and Tajima's estimates of
$\theta$ in the initial generation (Appendix~\ref{app:td}).  See
Suppl.~Fig.~\ref{fig:sfsts} for comparison to empirical values from
simulations. Similarly, we show (Appendix~\ref{app:h}), that the
dynamics of Fay and Wu's $H$ statistic~\cite{fay2000hitchhiking} are
directly related to expected value of the of Haplotype Allele
Frequency (HAF) score~\cite{ronen2015predicting}, and can be written
as a function of $\nu_t$ as follows:
\begin{equation}
  nH_t= \theta \nu_t \left(\frac{\nu_t+1}{2} -
    \frac{1}{(1-\nu_t)n+1}\right) + \theta
  (1-\nu_t)\left(\frac{n+1}{2n}-\frac{1}{(1-\nu_t)n+1}\right)
  \label{eq:ht}
\end{equation}	
In both cases, $\nu_t$ itself can be written as a function of $s,t$
(Suppl.~Eq.~\ref{eq:inf-pop}). This allows us to compute likelihood
functions $\Lc_S(s; \{D_t\})$ or $\Lc_S(s; \{H_t\})$. Then, a
likelihood ratio, similar to Eqns.~\ref{eq:mcts},~\ref{eq:hmmml}
provides a statistic for detecting selection in each window.

However, as $\nu_0$ and $D_0$ are often unknown in the sampling from
natural population experiments, and likelihoods $\Lc_S(.)$ are difficult to 
compute, we cannot directly use
Eqns.~\ref{eq:tdt},~\ref{eq:ht}. Instead, we heuristically aggregate statistics
throughout time to compute time-series score. See details in
Appendix~\ref{app:agg}.


\ignore{
\paragraph{$p$-value computation.}
By Wilks’ theorem~\cite{williams2001weighing}, the standard likelihood
ratio statistic (Eq.~\ref{eq:mcts}) is asymptotically distributed
according to $\Xc^2$. However, Feder et
al.~\cite{feder2014Identifying} point out that the empirical
distribution provide more accurate $p$-values than $\Xc^2$ when the
number of independent samples (replicates) is small.  Here we compute
$p$-value for the test statistic using the empirical distribution of
the negative controls.}
		


\subsection{Simulations}
We performed extensive simulations using parameters that have been
used for \dmel experimental
evolution~\cite{kofler2013guide}. See also Fig.~\ref{fig:ee} for
illustration. To implement real wold pool-seq experimental evolution, we 
conducted simulations as follows:
\begin{enumerate}[I.]
\item {\bf Creating initial founder line haplotypes.} Using
  \texttt{msms}~\cite{ewing2010msms}, we created neutral populations for $F$
  founding haplotypes with \emph{default} parameters \texttt{\$./msms
    <F> 1 -t <2$\mu$LNe> -r <2rNeL> <L>}, where $F=200$ is number of
  founder lines, $N_e=10^6$ is effective population size,
  $r=2*10^{-8}$ is recombination rate, $\mu=2\times 10^{-9}$ is
  mutation rate and $L=50K$ is the window size in base pairs which
  gives $\theta=2\mu N_eL=200$ and $\rho=2N_erL=2000$.
  
\item{\bf Creating initial diploid population.} To simulate
  experimental evolution of diploid organisms, initial haplotypes were
  first cloned to create $F$ diploid homozygotes. Next, each diploid
  individual was cloned $N/F$ times to yield diploid population of
  size $N$.

\item{\bf Forward Simulation.} We used forward simulations for
  evolving populations under selection. We note that all experiments
  denoted as \emph{soft-sweep} below, refer to selection acting upon
  standing variation. Given initial diploid population, position of
  the site under selection, selection strength $s$, number of
  replicates $R=3$, recombination rate $r=2\times10^{-8}$ and sampling
  times $\Tc=\{0,10,20,30,40,50\}$, \texttt{simuPop} was used to perform
  forward simulation and compute allele frequencies for all of the $R$
  replicates. To avoid spurious ``selection" simulation samples, simulation 
  results in which the frequency of favored allele is decreased are discarded.
  For hard sweep (respectively, soft sweep) simulations we
  randomly chosen a site with initial frequency of $\nu_0=0.005$
  (respectively, $\nu_0=0.1$) to be the favored allele.
  \item{\bf Sequencing Simulation.} Give allele frequency trajectories we 
  sampled depth of each site  identically and independently from 
  Poisson($\lambda$), where $\lambda \in \{30,100,\infty\}$ is the coverage for 
  the experiment. Once depth $d$ are drawn for the site with frequency $\nu$, 
  the number of derived allele read counts $c$ are sampled according to 
  Binomial$(d,\nu)$. For the experiments with finite depth the tuple $(c,d)$ is 
  the input data for each site, and for infinite depth experiments simply 
  allele frequency is given and Markov chain is evaluated for the HMM method, 
  i.e. $\Hc=\Mc$.
\end{enumerate}
We also conducted simulations to evaluate performance of the methods in the
cases which sample is taken from natural population. Sampling from natural 
populations differs from (controlled) experimental evolution in some important
ways. First, the time of onset of selection may not be known. Second, as the 
start of 
sampling can be any generation during selective sweep, mutations that arose 
after the onset of selection appear in data, and can have a nontrivial effect 
on SFS if sampling is started far after start of selection. In addition, larger 
time since the onset of selection can potentially improve accuracy of the 
site-localization task, due to recombination. Third, larger 
population sizes, which diminishes effect of genetic 
drift on one hand, and reduces the rate of evolution (i.e., larger fixation 
time) on the other hand\AI{is it true?}. \texttt{msms} was used to
forward-simulate a population with $N_e=10^4$, $\nu_0=10^{-4}$, and to
record SFS of a 50Kbp region (Fig.~\ref{fig:ee}A). The remaining
parameters were identical to controlled experimental evolution simulations.
