\section{Introduction}
Until very recently, biological data analysis has been considered processing a snapshot of data. However, the emergence of NGS and related technologies has made it possible to not only create larger datasets but also to measure multiple observations of the same quantity in the course of time. In many cases, such as population genetics, it is of the great interest to model the evolutionary process and make inferences, predictions and retrospective studies. Indeed, a random process is better explained by time series data than a single observation.

In addition to inexpensive data availability, over last two decades, a large amount of efforts is dedicated to High-Performance Computing (HPC), which re-popularized and re-branded computationally intensive algorithms such as Neural Networks. The first properly proposed neural network model to exploit full potential of multi layer neural networks published by \cite{deep-DR, deep-belief} and its spectacular performance on image processing problems immediately spawned the field of Deep Neural Networks (DNN), aka Deep Learning. Shortly after, DNNs went beyond the tasks that they are initially indented to accomplish \cite{deep-imagenet} and had breakthroughs in  time-series DNN models, aka RNNs, such as generative models \cite{deep-generative}, speech processing \cite{deep-speech} etc. 

In this paper we aim to use the tools and machinery that has been developed for RNNs, to model times-series biological data. In particular, we consider the population genetics problem of finding loci (locus) under selection given observation of allele frequency of a population in different generations of a Wright-Fisher model. This problem has been previously treated by using Gaussian Processes \cite{EnadR-GP}, spectral methods \cite{EandR-spectral}



