\section{Introduction}

Genetic adaptation is \emph{the} central evolutionary process and is
at the core of some of the greatest challenges facing humanity. HIV
would likely cause nothing more than harmless fever without the
ability of the virus to adapt and eventually destroy the immune
system. Cancer would be much more straightforward to treat if not for
tumor's ability to adapt to anti-cancer drugs. Malaria could be
treated with cheap drugs such as quinine instead of being one of the
world's worst killers. Crop pests would be manageable with small doses
of safe insecticides instead of requiring applications of ever
increasing amounts of a diverse array of powerful chemicals. In both
disease and agriculture, we find ourselves in an arms race due to the
ability of organisms to adapt.

Adaptation leaves a variety of detectable signatures in
genomes~\cite{Akey09, Kreitman00, MesserAndPetrov13, Nielsen05,
  SabetiEtAl06}. The rapid expansion of adaptive alleles in
populations leads to both an excess of functional changes between
species and distortions in polymorphism patterns known as selective
sweeps~\cite{Nielsen05}. The signatures of selective sweeps, which
include reduction of levels of polymorphism and the presence of too
many rare alleles~\cite{Nielsen05, Przeworski02}, have been the basis
for assessing genomic patterns in many genome scans for
adaptation. Classical tests such as the well-known Tajima's \emph{D}
statistic~\cite{Tajima89} that rely on the \emph{site-frequency
  spectrum}---a list of counts of the numbers of genomic sites in a
region with each of the different possible allele frequencies---have
been among the first steps in genomic detection of selection, but they
have largely been chosen in a non-systematic manner to identify quite
specific rather than general signatures of selection, and they have
often faced the problem of confounding of positive selection with
demographic processes~\cite{PtakAndPrzeworski02, RamosOnsinsAndRojas}.
Part of the challenge is that studies on natural populations are
conducted on a extant populations, a single window in a complex,
uncontrolled process.

Experimental evolution (including evolve and re-sequence paradigms)
have become increasingly popular as a complementary tool to understand
the forces of selection, allowing for controlled environments,
specific selection constraints. Examples of experimental evolution
studies abound in sexual populations, particularly in fruitfly. Burke
et al.~\cite{Burke2010} evolved flies for over $600$ generations under
selection for accelerated development, and noted that evolution for
sexual populations is very different from those of asexual populations
(e.g., Lenski~\cite{}): the effects of clonal interference is slower
due to recombination that allows for the incorporation of multiple
beneficial alleles, but there are fewer, unconditionally advantageous
alleles that arise at the onset of selection. Rose and
Colleagues~\cite{Rose} created $200$ experimentally evolved
populations selected for different traits, starting with $10$ initial
populations. Zhou et al. evolved flies to adapt to low oxygen
environment (hypoxia) for over $300$ generations, and identified many
genes involved in hypoxia tolerance~\cite{}.  Instead, they observed
incomplete fixation (`soft-sweeps') due in part to standing variation,
changing selection coefficients, and small fitness effects.

Much like in natural populations, many of these studies also
sequenced/genotyped only the latest population. However, the emergence
of NGS and related technologies has made it feasible to sequence the
evolving population at multiple time points in its evolution. At the
same time, for small organisms where a single animal does not provide
enough DNA, it is more effective to pool together individuals in the
population at a particular time point, in a single sequencing
experiment. Thus, instead of individual genotypes, we obtain
frequencies of the derived allele at all loci at different time
points. Methods for analyzing such time series pooled sequence data
are only just being developed.

In this paper we aim to use the tools and machinery that has been
developed for RNNs, to model times-series biological data. In
particular, we consider the population genetics problem of finding
loci (locus) under selection given observation of allele frequency of
a population in different generations of a Wright-Fisher model. This
problem has been previously treated by using Gaussian Processes
\cite{Terhorst15}, spectral methods \cite{Song14}



