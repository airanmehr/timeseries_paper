\section{Materials and Methods}
\label{sec:method}
Consider a panmictic diploid population with fixed size of $N$
individuals.  Let $\bm{ \nu} =\{\nu_t\}_{t\in\Tc}$ be frequencies of
the derived allele at generations $t\in\Tc$ for a given variant, where
at generations $\Tc= \{\tau_i: 0\le \tau_0<\tau_1\ldots< \tau_T\}$
samples of $n$ individuals are chosen for pooled sequencing. The
experiment is replicated $R$ times. We denote allele frequencies of
the $R$ replicates by the set $\{\bm{\nu}\}_R$.  To identify the genes
and variants that are responding to selection pressure, we use the
following procedure:
\begin{enumerate}
\item {\bf Estimating population size.} The procedure starts by
  estimating the effective population size, $\hN$, under the
  assumption that much of the genome is evolving neutrally.
\item {\bf Estimating selection parameters.} For each polymorphic
  site, selection and dominance parameters $s,h$ are estimated so
  as to maximize the likelihood of the time series data, given $\hN$.
\item {\bf Computing likelihood statistics.} For each variant, a
  log-odds ratio of the likelihood of selection model ($s>0$) to the
  likelihood of neutral evolution/drift model is computed. Likelihood
  ratios in a genomic region are combined to compute the \comale\
  statistic for the region.
\item {\bf Hypothesis testing.} An empirical null distribution of the
  \comale\ statistic is calculated using genome-wide drift
  simulations, and used to compute $p$-values and thresholds for a
  specified FDR. We perform single locus hypothesis testing within
  selected regions to identify significant variants and report genes
  that intersect with the selected variants.
\end{enumerate}
These steps are described in detail below.
\subsection{Estimating Population Size}
Methods for estimating population sizes from temporal neutral
evolution data have been
developed~\cite{williamson1999using,anderson2000monte,
  bollback2008estimation, Terhorst2015Multi,jonas2016estimating}.
Here, we aim to extend these models to explicitly model the sampling
noise that arise in pool-seq data. Specifically, we model the
variation in sequence coverage over different locations, and the noise
due to sequencing only a subset of the individuals in the population.
In addition, many existing
methods~\cite{bollback2008estimation,feder2014Identifying,topa2015gaussian,Terhorst2015Multi}
are designed for large populations, and model frequency as a
continuous quantity. We observed that using Brownian motion to model
frequency drift may be inadequate for small populations, low starting
frequencies and sparse sampling (in time), factors that are common in
experimental evolution (see Results, \ref{fig:power}A-C, and
\ref{fig:markov}). To this end, we model the Wright-Fisher Markov
process for generating pool-seq data~(\ref{proc:arya}) via a
{discrete} HMM (~\ref{fig:1}-B). We start by computing a likelihood
function for the population size given neutral pool-seq data.


\paragraph{Likelihood for Neutral Model.}
We model the allele frequency counts $2N\nu_t$ as being sampled from a
Binomial distribution. Specifically,
\begin{eqnarray*} 
  \nu_0 &\sim& \pi,\\
  2N\nu_t|\nu_{t-1} &\sim& \bino(2N,\nu_{t-1}) 
\end{eqnarray*}
where $\pi$ is the global distribution of allele frequencies in the
base population. 
Note that  $\pi$  
depends on the
demographic history of the founder lines and can be estimated from site 
frequency spectrum(see~\ref{fig:sfs}) of the initial population. For notational 
convenience, henceforth we omit the dependence of likelihoods to the 
parameter $\pi$.

To estimate frequency after $\tau$ transitions, it is enough to
specify the $2N\times2N$ transition matrix $P^{(\tau)}$, where
$P^{(\tau)}[i,j]$ denotes probability of change in allele frequency
from ${i}/{2N}$ to ${j}/{2N}$ in $\tau$ generations:
\beq
  P^{(1)}[i,j] &= \pr\left(\nu_{t+1}=\frac{j}{2N} \left| \right. 
  \nu_{t}=\frac{i}{2N}\right)\\
  &={2N \choose j} s  \nu_{t}^j
  (1-\nu_{t})^{2N-j}, \label{eq:P1}
\eeq
\beq
  P^{(\tau)} =   P^{(\tau-1)}P^{(1)} \label{eq:Pt}
\eeq
Furthermore, in an E\&R experiment, $n\le N$ individuals are randomly
selected for sequencing. The {sampled allele frequencies},
$\{y_{t}\}_{t\in\Tc}$, are also Binomially distributed \beq 2ny_{t}
\sim \text{Binomial}(2n,\nu_t) \eeq We introduce the $2N\times2n$
sampling matrix $Y$, where $Y[i,j]$ stores the probability that the
sample allele frequency is ${j}/{2n}$ given that the true allele
frequency is ${i}/{2N}$.


We denote the pool-seq data for that variant as $\{x_t = \langle
c_t,d_t \rangle\}_{t\in\Tc}$ where $d_t, c_t$ represent the coverage,
and the read count of the derived allele, respectively. Let
$\{\lambda_t\}_{t\in\Tc}$ be the sequencing coverage at different
generations. Then, the observed data are sampled according to
\begin{equation} d_t \sim \poiss(\lambda_t), \hspace{1in} c_t \sim
\bino(d_t,y_t) 
\end{equation}
The emission probability for a observed tuple $x_t=\langle d_t,
c_t\rangle $ is 
\begin{equation} {\bf e}_{i}(x_t) = {d_t \choose c_t}
\left(\frac{i}{2n} \right)^{c_t}\left (1- \frac{i}{2n}
\right)^{d_t-c_t} .  
\end{equation}
For $1\le t\le T, 1\le j\le 2N$, let $\alpha_{t,j}$ denote the
probability of emitting $x_1,x_2,\ldots,x_t$ and reaching state $j$ at
$\tau_t$. Then, $\alpha_{t}$ can be computed using the
forward-procedure~\cite{durbin1998biological}:
\begin{align}
	\alpha_t^T&=\alpha_{t-1}^T P^{(\delta_t)}\text{diag}(Y\bfe(x_t))
	\label{eq:hmm}
\end{align}
where $\delta_t=\tau_t-\tau_{t-1}$. The joint likelihood of the
observed data from $R$ independent observations is given by
\beq
\Lc(N|\{\bm{x}\}_R, 
n)&=\prod_{r=1}^R\Lc(N|\bm{x}^{(r)},n)=\pr(\{\bm{x}\}_R|N,n)\\& =	
	\prod_{r=1}^R \sum_i\alpha_{T,i}^{(r)}
	\label{eq:hmmlik}
\eeq
where $\bm{x}=\{x_t\}_{t\in \Tc}$. The graphical model and the generative 
process for which data is being generated is depicted in~\ref{fig:1}-B 
and~\ref{proc:arya}, respectively.

Finally, the last step is to compute an estimate $\hN$ that maximizes
the likelihood of all $M$ variants in whole genome. Let
$\bm{x}_i^{(r)}$ denote the time-series data of the $i$-th variant in
replicate $r$. Then,
\begin{equation}
 \hN =
\underset{N}{\arg \max} \prod_{i=1}^M  \prod_{r=1}^R\Lc(N|\bm{x}_i^{(r)})
\label{eq:mlen}
\end{equation}

\subsection{Estimating Selection Parameters}

\paragraph{Likelihood for Selection Model.}
Assume that the site is evolving under selection constraints $s\in
\Rbb$, $h\in \Rbb_+$, where $s$ and $h$ denote selection strength and 
dominance parameters ,
respectively. By definition, the relative fitness values of genotypes
0$|$0, 0$|$1 and 1$|$1 are given by $w_{00}=1$, $w_{01}=1+hs$ and
$w_{11}=1+s$.  Then, $\nusp$, the frequency at time
$\tau_{t}+1$ (one generation ahead), can be estimated using: 
\beq 
\hat{\nu}_{t^+} =
\mathbb{E}[\nusp|s,h,\nu_t]&=\frac{w_{11}\nu_t^2 +
  w_{01}\nu_t(1-\nu_t)}{w_{11}\nu^2_t + 2w_{01}\nu_t(1-\nu_t) +
  w_{00}(1-\nu_t)^2}\\
&=\nu_t+\frac{s(h+(1-2h)\nu_t)\nu_t(1-\nu_t)}{1+s\nu_t(2h+(1-2h)\nu_t))}.
  \label{eq:transition}
\eeq
The machinery for computing likelihood of the selection parameters is 
identical to that of population size, except for transition matrices. Hence, here 
we only describe the definition transition matrix $Q_{s,h}$ of the selection 
model.
Let $Q^{(\tau)}_{s,h}[i,j]$ denote the
probability of transition from ${i}/{2N}$ to ${j}/{2N}$ in
$\tau$ generations, then (See~\cite{Ewens2012Mathematical}, Pg.~24, 
Eqn.~$1.58$-$1.59$):
\beq
  Q^{(1)}_{s,h}[i,j] &= \pr\left(\nusp=\frac{j}{2N} \left\lvert
      \nu_{t}=\frac{i}{2N};s,h,N \right .\right)\\
  &    ={2N \choose j}
  \hat{\nu}_{t^+}^{j} (1-\hat{\nu}_{t^+})^{2N-j}\label{eq:Q1}
  \eeq
  \beq
  Q^{(\tau)}_{s,h} &= Q^{(\tau-1)}_{s,h}Q^{(1)}_{s,h}\label{eq:Qt}  
\eeq
The maximum likelihood estimates are given by
\beq
\hs,\hh = \underset{s,h}{\arg \max} \prod_{r=1}^R \Lc(s,h|\bm{x}^{(r)},\hN) 
\label{eq:mlesh}
\eeq

Using grid search, we first estimate $N$ (Eq.~\ref{eq:mlen}), and
subsequently, we estimate parameters $s,h$ 
(Eq.~\ref{eq:mlesh},~\ref{fig:slikes}). By
broadcasting and vectorizing the grid search operations across all
variants, the genome scan on millions of polymorphisms can be done in
significantly smaller time than iterating a numerical optimization
routine for each variant(see Results and \ref{fig:runTime}).
\subsection{Empirical Likelihood Ratio Statistics}
The  likelihood
ratio statistic for testing directional selection, to be computed for each variant, 
is given by
\beq
	H &= -2 \log 
	\left(\frac{\Lc(\bar{s},0.5|\{\bm{x}\}_R,\hN)}{\Lc(0,0.5|\{\bm{x}\}_R,\hN)}\right),\\
	\label{eq:ELRS}
\eeq
where $\bar{s} = \underset{s}{\arg \max} \prod_{r=1}^R 
 \Lc(s,0.5|\bm{x}^{(r)},\hN)$. Similarly we can define a test statistic for testing 
 if selection is dominant by
\beq
 D &= -2 \log 
 \left(\frac{\Lc(\hs,\hh|\{\bm{x}\}_R,\hN)}{\Lc(\bar{s},0.5|\{\bm{x}\}_R,\hN)}\right).
 \eeq



 While extending the single-locus WF model to a multiple linked-loci
 can improve the power of the model~\cite{Terhorst2015Multi}, it is
 computationally and statistically expensive to compute exact
 likelihood. In addition, computing linked-loci joint likelihood requires  
 haplotype resolved data, which pool-seq
 does not provide. Here, similar to Nielsen~\emph{et
   al}~\cite{nielsen2005genomic}, we calculate \emph{composite likelihood
 ratio} score for a genomic region.
\begin{equation}
\Hc = \frac{1}{|L|}\sum_{\ell \in L} H_\ell.
\label{eq:pihmm}
\end{equation}
where $L$ is a collection of segregating sites and $H_\ell$ is the
likelihood ratio score based for each variant $\ell$ in $L$.  The
optimal value of the hyper-parameter $L$ depends upon a number of
factors, including initial frequency of the favored allele,
recombination rates, linkage of the favored allele to neighboring
variants, population size, coverage, and time since the onset of
selection (duration of the experiment). In~\ref{sec:winSize}, we
provide a heuristic to compute a reasonable value of $L$, based on
experimental data. 

We work with a normalized value of $\Hc$, given by
\begin{equation} \Hc_i^*=\frac{\Hc_i-\mu_\Cc}{\sigma_\Cc},
\hspace{0.5in} \forall i \in \Cc,
\end{equation} 
where $\mu_\Cc$ and $\sigma_\Cc$ are the mean and standard deviation
of $\Hc$ values in a large region $\Cc$. We found different
chromosomes to have different distribution of $\Hc_i$ values, and
therefore decided to use single chromosomes as $\Cc$.
\subsection{Hypothesis Testing}
\paragraph{Single-Locus tests.}
Under neutrality, Log-likelihood ratios can be approximated by $\Xc^2$
distribution~\cite{williams2001weighing}, and $p$-values can be
computed directly. However, Feder \emph{et
  al.}~\cite{feder2014Identifying} showed that when the number of
independent samples (replicates) is small, $\Xc^2$ is a crude
approximation to the true null distribution and results in more false
positive.  Following their suggestion, we first compute the empirical
null distribution using simulations with the estimated population size
(See~\ref{proc:arya}). The empirical null distribution of statistic
$H$ is used to compute $p$-values as the fraction of null values that
exceed the test score.  Finally, we use Storey and Tibshirani's
method~\cite{storey2003statistical} to control for False Discovery
Rate in multiple testing.


\paragraph{Composite likelihood tests.}

Similar to single-locus tests, we compute the null distribution of the
$\Hc^*$ statistic using whole-genome simulations with the estimated
population size, and subsequently compute FDR. The simulations for
generating the null distribution of $\Hc^*$ are described next. 

\subsection{Simulations}\label{sec:sims}
We use the same simulation procedure for two purposes. First, we use
them to test the power of \comale\ against other methods in small
genomic windows. Second, we use the simulations to generate the
distribution of null values for the statistic to compute empirical
$p$-values. We mainly chose parameters that are relevant to \dmel
experimental evolution~\cite{kofler2013guide}. See also \ref{fig:1}-A
for illustration. 
\begin{enumerate}
\item {\bf Creating initial founder line haplotypes.} Using
  \texttt{msms}~\cite{ewing2010msms}, we created neutral populations
  for $F$ founding haplotypes with command \texttt{\$./msms <F> 1 -t
    <2$\mu WN_o$> -r <$2rWN_o$> <W>}, where $F=200$ is number of 
    founder
  lines, $N_o=10^6$ is effective founder population size,
  $r=2\times10^{-8}$ is recombination rate, $\mu=2\times 10^{-9}$ is
  mutation rate. The window size $W$ is used to compute $\theta=2\mu
  N_oW$ and $\rho=2N_orW$. We chose $W=50$Kbp for simulating
  individual windows for performance evaluations, and $W=20$Mbp for
  simulating \dmel chromosomes for $p$-value computations.
  
\item{\bf Creating initial diploid population.} An initial set of
  $F=200$ haplotypes was created from step I, and duplicated to create
  $F$ homozygous diploid individuals to simulate generation of inbred
  lines. $N$ diploid individuals were generated by sampling with
  replacement from the $F$ individuals.

\item{\bf Forward Simulation.} We used forward simulations for
  evolving populations under selection. We also consider selection
  regimes which the favored allele is chosen from standing variation
  (not \emph{de novo} mutations). Given initial diploid population,
  position of the site under selection, selection strength $s$, number
  of replicates $R=3$, recombination rate $r=2\times10^{-8}$ and
  sampling times $\Tc=\{0,10,20,30,40,50\}$,
  \texttt{simuPop}~\cite{peng2005simupop} was used to perform forward
  simulation and compute allele frequencies for all of the $R$
  replicates.  For hard sweep (respectively, soft sweep) simulations
  we randomly chose a site with initial frequency of $\nu_0=0.005$
  (respectively, $\nu_0=0.1$) to be the favored allele. For generating
  the null distribution with drift for $p$-value computations, we used
  this procedure with $s=0$.

\item{\bf Sequencing Simulation.} Given allele frequency trajectories
  we sampled depth of each site in each replicate identically and
  independently from Poisson($\lambda$), where $\lambda \in
  \{30,100,300\}$ is the coverage for the experiment. Once depth $d$
  is drawn for the site with frequency $\nu$, the number of reads $c$
  carrying the derived allele are sampled according to
  Binomial$(d,\nu)$. For experiments with finite depth the tuple
  $\langle c,d\rangle$ is the input data for each site.
\end{enumerate}
