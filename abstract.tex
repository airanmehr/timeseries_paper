\begin{abstract}
	Experimental evolution (EE) studies are powerful tools for observing
	molecular evolution ``in-action'' in populations sampled in
	controlled and natural environments. The advent of sequencing
	technologies has made whole-genome and whole-population sampling
	possible even for eukaryotic organisms with large genomes, and
	allowed us to locate the genes and variants responsible for genetic
	adaptation. While many computational tests have been developed for
	detecting regions under selection, they are mainly designed for
	static (single time) data, and work best when the favored allele is
	close to fixation. 
	
	EE studies provide samples over multiple time points, underscoring
	the need for tools that can exploit the data. At the same time, EE
	studies are constrained by the limited time span since onset of
	selection, depending upon the generation time for the organism. This
	constraint impedes adaptation and optimization studies, as the
	population can only be propagated for a small number of generations,
	relative to the fixation time of the favored allele.  Moreover,
	sequencing depths of pooled experiments vary across replicates and
	time points, leading to ascertainment bias.
	
	In this article, we directly address these issues while developing
	tools for identifying selective sweep in short-term and
	pool-sequenced experimental evolution of sexual organisms and
	propose Composite Of MArkovian Likelihoods for Experimental
	evolution (\comale) statistic. Extensive simulations show that
	\comale\ achieves higher detection power over a wide range of
	parameters, including both soft and hard sweep scenarios, and is
	many orders of magnitude faster than other methods, making genomic
	scans feasible.  We apply the \comale\ statistic to the controlled
	experimental evolution of D. melanogaster to detect adaptive
	genes/alleles under alternating cold and hot temperatures, and
	identify many genes that are adapting to the selection constraints.
\end{abstract}