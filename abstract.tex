\begin{abstract}
  Experimental evolution (EE) studies are powerful tools for observing
  molecular evolution ``in-action'' from populations sampled in
  controlled and natural environments. The advent of next generation
  sequencing technologies has made whole-genome and whole-population
  sampling possible, even for eukaryotic organisms with large genomes,
  and allowed us to locate the genes and variants responsible for
  genetic adaptation. While many computational tests have been
  developed for detecting regions under selection, they are mainly
  designed for static (single time) data, and work best when the
  favored allele is close to fixation.
	
  EE studies provide samples over multiple time points, underscoring
  the need for tools that can exploit the data. At the same time, EE
  studies are constrained by the limited time span since onset of
  selection, depending upon the generation time for the organism. This
  constraint impedes adaptation and optimization studies, as the
  population can only be propagated for a small number of generations
  relative to the fixation time of the favored allele. Moreover,
  coverage in pool-sequenced experiments varies across replicates and
  time points, leading to heterogeneous ascertainment bias in
  measuring population allele frequency across different measurements.
	
  In this article, we directly address these issues while developing
  tools for identifying selective sweep in pool-seq EE of sexual
  organisms and propose Composition of Likelihoods for
  Evolve-And-Resequence experiments (\comale). Extensive simulations
  show that \comale\ achieves higher power in detecting and localizing
  selection over a wide range of parameters. In contrast to existing
  methods, the \comale\ statistics are robust to variation of
  coverage. \comale\ also provides robust estimates of model
  parameters, including selection strength and overdominance, as
  byproduct of the statistical test, while being orders of magnitude
  faster than existing methods. Finally, we apply the \comale\
  statistic to \datadm\ and discover an enrichment of adaptation in
  genes with peroxidase activity.
\end{abstract}