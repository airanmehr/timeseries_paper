\section{Results}
\paragraph{Modeling allele frequency trajectories in finite populations.} 
We first tested the goodness of fit of the discrete-time Markov chain
versus continuous-time Brownian motion in modeling allele frequency
trajectories in finite populations, under different sampling schemes
and starting frequencies.  For this purpose, we conducted $150$K
simulations with two time samples $\Tc=\{0,\tau\}$ where $\tau\in
\{1,10,100\}$ is the parameter controlling the frequency of sampling
in time.  In addition, we repeated simulations for different values of
starting frequency $\nu_0\in\{0.005,0.1\}$ (i.e., hard and soft sweep)
and selection strength $s\in\{0,0.1\}$ (i.e., neutral and
selection). Then, given initial frequency $\nu_0$, we computed
expected distribution of the frequency of the next sample $\nu_\tau$
under two models and compared them with empirical distributions
calculated from simulated data.  Fig.~\ref{fig:markov}A-F shows that
Brownian motion (Gaussian approximation) is inadequate when $\nu_0$ is
far from $0.5$, and when sampling times are sparse ($\tau>1$). If the
favored allele arises from standing variation in a neutral population,
it is unlikely to have frequency close to $0.5$, and the starting
frequencies are usually much smaller (see
Suppl. Fig.~\ref{fig:sfs}). Moreover, in typical \dmel experiments for
example, sampling is sparse. Often, the experiment is designed so that
$10\le\tau\le100$~\cite{kofler2013guide, orozco2012adaptation,
  zhou2011experimental}.

In contrast to the Brownian motion results, Fig.~\ref{fig:markov}A-I
also shows that Markov chain predictions (Eq.~\ref{eq:mkvs}) are
highly consistent with empirical data for a wide range of simulation
parameters for both selection and neutral evolution.

\paragraph{Detection Power.} 
We compared the performance of \comale\ against other methods for
detecting selection. Define power as the fraction of true-positives
identified with false-positive rate $\le 0.05$
(Suppl.~Fig.~\ref{fig:powerROC}). We also took into account of
ascertainment bias in our comparisons, by evaluating each simulation
for different coverage values sampled from a Poisson distribution with
mean $\lambda\in \{30,100,\infty\}$.  To evaluate power for each
configuration (specified with values for selection coefficient $s$,
starting allele frequency $\nu_0$ and coverage $\lambda$), we
conducted $1000$ simulations, half of which modeled neutral evolution
and the rest modeled positive selection.

Before comparing against other methods, we first evaluated the use of
\comale\ with different percentile-cutoffs $\pi$ (Eq.~\ref{eq:pihmm})
in computing composite statistics of a region. For each configuration,
we computed average Power for $s\in\{0.025,0.05,0.075,0.1\}$, using
$\Hc_\pi,\Hc^{+}_\pi$. We computed the optimal value of $\pi$ using a
line-search. Fig.~\ref{fig:clrq} reveals several important trade-off
between $\pi$, initial frequency, and coverage.
\begin{packed_itemize}
\item $\Hc^{+}_\pi$ consistently achieves a high value for $\pi=0$,
  and in the absence of knowledge of the selection regime or the
  ancestral allele is unknown, $\Hc^{+}_0$ is a good statistic to use.
\item In every scenario tested,
  $\max_{\pi}\{\mbox{Power}(\Hc_\pi)\}\ge
  \max_{\pi}\{\mbox{Power}(\Hc^{+}_\pi)\}$, suggesting that it helps
  to choose $\Hc_{\pi}$ with the optimum value of $\pi$, if the
  selection regime is well-understood.
\item In soft sweep, relative to hard sweep, it helps to choose a
  higher value of the cut-off $\pi$. This is consistent with LD
  between the favored site and other sites being lower for
  soft sweep. For instance, in soft sweep with infinite coverage
  (Fig.~\ref{fig:clrq}F), optimum is gained at $\pi=100$, equivalent
  to considering the highest scoring site.
\item When coverage is low (Fig.~\ref{fig:clrq}A,D), it helps to
  accumulate evidence from multiple sites, and the best results are
  achieved for lower values of $\pi$.
\end{packed_itemize}
In the following tests, we fix the value of $\pi=0.95,0.98, 0.99$ for
$\lambda=30,100,\infty$, respectively, and simplify notation by
denoting the optimum score as $\Hc$. For example, when $\lambda=30$,
$\Hc$ corresponds to $\Hc_{0.95}$. In addition, we use $\Hc^{+}$ to
denote $\Hc^{+}_0$. We also compared $\Mc$ (Markov likelihoods) versus
$\Hc$. As shown in Fig.~\ref{fig:powerCLR}, $\Hc$ has better power for
low coverage ($\lambda=30$) compared to $\Mc$ decays.


\ignore{
, and modified likelihood ratio
$H$ to compute \comale\ statistics.
}

Finally, we compared the power of \comale\ with Gaussian process
(GP)~\cite{Terhorst2015Multi}, FIT~\cite{feder2014Identifying}, and
CMH~\cite{agresti2011categorical} statistics.  As CMH only takes read
count data, here we used $\lambda=300$ to implement infinite coverage
scenario. All methods other than \comale\ and CMH convert read counts
to allele frequencies prior to computing the test statistic.  \comale\
shows the highest power in all cases and the power stays relatively
high even for low coverage (Fig.~\ref{fig:power} and
Table~\ref{XXX}). In particular, the difference in performance of
\comale\ with other methods is pronounced when starting frequency is
low (hard sweep). This stems from the fact that favored alleles with
low frequency alleles might be missed by low coverage sequencing, and
the contribution of other sites becomes more important. The power of
\comale\ improves consistently with increasing values of $s$. We note
that methods using only two time points, such as CMH, do relatively
well for high selection values and high coverage. However, time-series
data can be used to get estimates of selection parameters $s,h$ (see
below), and our results (Fig.~\ref{fig:power}B,C) suggest that taking
many samples with lower coverage is preferable to sparse sampling with
higher coverage.


\ignore{ Moreover, the observation of
  Fig.~\ref{fig:markov} is revisited here, approximate continues
  models (GP and FIT) perform poorly when starting frequency is low
  (Fig.~\ref{fig:power}A-C).  }

\paragraph{SFS for Detection in Natural Samples.} We did not show the
SFS based statistics in Fig.~\ref{fig:power} as they did not perform
better than random. In many experimental evolution settings, we sample
a restricted set of $F$ founder lines, where $F<<N_e$
(Fig.~\ref{fig:ee}B) and inbred during the experiment. This creates a severe 
bottleneck, confounding
SFS. Suppl.~Fig.~\ref{fig:bottleneck} demonstrates the effect of
experimental evolution on different SFS statistics under neutral
evolution for 1000 simulations. A second problem with using SFS for
experimental evolution is that the sampling starts right after the
onset of experimentally induced selection, and the favored allele may
not reach high enough frequency to modify the site frequency spectrum
(Suppl. Fig.~\ref{fig:sweep}).

However, in experiments involving naturally occurring populations,
even if the span of the time-series is small, the onset of selection
might occur many generations prior to sampling. To test performance of
SFS-based statistics in natural evolution, we conducted 200 (100
neutral and 100 sweep) forward simulations for different values of
$s,\lambda$ using $N_e=10K$ and accumulating new mutations. The start
of sampling was done at a randomly picked time subsequent to the onset
of selection in two distinct scenarios. Let $t_{\nu=x}(s,N_e)$ denote
the expected time (in generations) required to reach carrier frequency
$x$ in a hard sweep and $U[a,b]$ denote discrete uniform distribution
in the interval $[a,b]$. First we considered the case when start of
sampling is chosen throughout the whole sweep. i.e., $\tau_1 \sim
U\left[1,t_{\nu=1}(s,N_e)\right]$ (Fig.~\ref{fig:powerSFS}A). Next, we
considered sampling start time chosen nearer to fixation of the
favored allele, i.e., $\tau_1 \sim
U\left[t_{\nu=0.9}(s,N_e),t_{\nu=1}(s,N_e)\right]$
(Fig.~\ref{fig:powerSFS}B). In both scenarios, sampling was done over
$5$ time points within $50$ generations of $\tau_1$. We compared
$\Hc$, GP, FIT with both static and dynamic SFS based statistics of
SFSelect and Tajima's D. Fig.~\ref{fig:powerSFS}A shows that SFS based
statistics are outperformed by single locus and CLR methods. However,
when sampling is performed close to fixation, i.e., when the favored
allele has frequency of 0.9 or higher, SFS based statistics perform
significantly better than GP, FIT and $\Hc$
(Fig.~\ref{fig:powerSFS}b). Moreover, dynamic SFS statistics
outperform static SFS statistics, demonstrating that in these regimes,
SFS based statistics could be used to detect selection.


\paragraph{Site-identification.}
Localizing the favored site is a nontrivial task. We used the simple
approach of ranking each site in a region detected as being under
selection. The sites were ranked according to the likelihood ratio
scores (Eqns.~\ref{eq:mcts},~\ref{eq:hmmml}). For each setting of
$\nu_0$ and $s$, we conducted $500$ simulations and computed the rank
of the favored mutation in each simulation. The cumulative
distribution of the rank of the favored allele in 500 simulation for
each setting (Fig.~\ref{fig:rank}) shows that \comale\ outperforms
other statistics. We also compared each method to see how often it
ranked the favored site in as the top ranked site
(Table~\ref{tab:rank}A-B), among the top 10 ranked sites
(Table~\ref{tab:rank}C-D), and among the top 50
(Table~\ref{tab:rank}E-F) ranked sites. In the 1150 variants tested,
\comale\ performed consistently better than other methods in all of
these measures.

An interesting observation is the contrast between site-identification
and detection. When selection coefficient is high, detection is easier
(Fig.~\ref{fig:power}A-F), but site-identification is harder due to
the high LD between hitchhiking sites and the favored allele
(Table~\ref{tab:rank}A-F).  Moreover, site-identification is harder in
hard sweep scenarios relative to soft sweeps. For example, when
coverage $\lambda=100$ and selection coefficient $s=0.1$, the
detection power is 80\% for hard sweep, but 100\% for soft sweep
(Fig.~\ref{fig:power}B-E). In contrast, the favored site was ranked as
the top in 14\% of hard sweep cases, compared to and 95\% of soft
sweep simulations (Table~\ref{tab:rank}A-B).  Our results are
consistent with previous
studies~\cite{long2013massive,tobler2014massive}. See 
Appendix~\ref{app:ld} for a detailed explanation.
 
\paragraph{Estimating Parameters.}
\comale\ computes the selection parameters $\hat{s}$ and $\hat{h}$ as
a byproduct of the hypothesis testing. We computed bias of selection
fitness ($s-\hat{s}$) and over dominance ($h-\hat{h}$) for of \comale\
and GP in each setting. The distribution of the error (bias) for
100$\times$ coverage is presented in Fig.~\ref{fig:bias100} for
different configurations.
Suppl.~Fig.~\ref{fig:bias30},~\ref{fig:biasinf} provide the
distribution of estimation errors for 30$\times$, and infinite
coverage, respectively.  For hard sweep, \comale\ provides estimates
of $s$ with lower variance of bias (Fig.\ref{fig:bias100}A). In soft
sweep, GP and \comale\ both provide unbiased estimates with low
variance (Fig.~\ref{fig:bias100}B). Fig.~\ref{fig:bias100}C-D shows
that \comale\ provides unbiased estimates of $h$ as well.

\paragraph{Running Time.}
As \comale\ does not compute exact likelihood of a region (i.e., does
not explicitly model linkage between sites), the complexity of
scanning a genome is linear in number of polymorphisms.  Calculating
score of each variant requires $\Oc(TR)$ and $\Oc(TRN^2)$ computation
for $\Mc$, and $\Hc$, respectively. However, most of the operations
are can be vectorized for all replicates to make the effective running
time for each variant, $\Oc(T)$ and $\Oc(TN)$, respectively.  We
conducted $1000$ simulations and measured running times for \comale-M,
\comale-H, FIT, CMH and GP with different number of linked-loci.  Our
analysis reveals (Fig.~\ref{fig:runTime}) that \comale\ is orders of
magnitude faster than GP, and comparable to FIT. While slower than CMH
on the time per variant, the actual running times are comparable after
vectorization and broadcasting over variants (see below).

These times can have a practical consequence. For instance, to run GP
in the single locus mode on the entire pool-seqed \dmel genome of a
small sample (1.5M variant sites), it would take 1444 CPU-hours
($\approx$ 1 CPU-month). In contrast, after vectorizing and
broadcasting operations for all variants operations using
\texttt{numba} package, \comale\ takes less than an hour to perform an
scan, including precomputation. \VB{CMH took XXX minutes}

\subsection{Analysis of a \dmel EE experiment}\label{sec:dmel}
We applied \comale\ to the \data~\cite{orozco2012adaptation}, where 3
replicate samples were chosen from a population of \dmel for 59
generations under alternating 12-hour cycles of hot (28$^{\circ}$C)
and cold (18$^{\circ}$C) temperatures and sequenced.  In this dataset,
sequencing coverage is different across replicates and generations
(see Fig. S2 of~\cite{Terhorst2015Multi}) which makes variant depths
highly heterogeneous (Suppl.
Figs.~~\ref{fig:depth},~\ref{fig:depthHetero}). Filtering low coverage
sites from data can dramatically reduce data available for
analysis. For example, by setting minimum read depth at a site to be
$30$, allowing the site to be retained only if the depth for all time
points, and all replicates exceeded $30$, the number of sites analyzed
would drop from from 1,544,374 to 10,387. The \comale\ ($\Hc^{+}$)
statistic automatically handles such a heterogeneity in HMM (see
Fig.~\ref{fig:stateConditional}). We computed the $\Hc^{+}$ statistic
for sliding window of 50Kbp with steps of 10Kbp over the whole
genome. We filtered out genomic regions containing fewer than $500$
SNPs, as those regions were close to the centromere or
telomere. Fig.~\ref{fig:manhattancutoffed} shows a Manhattan plot of
scores. Our results are consistent with previous studies in showing an
over-representation of significant variants on Chromosome
3R~\cite{orozco2012adaptation} and identifying intervals under
selection~\cite{Terhorst2015Multi}.

The 1\%-ile cut-off (dashed line in Fig.~\ref{fig:manhattancutoffed})
reveals $21$ distinct intervals (Table~\ref{tab:intervals}) spanning
$243$ of the $13,965$ genes that encoded variants. We found $36$ GO
terms associated with ``Biological Process'' to be over-represented in
the $243$ genes using the Fisher exact test for
significance. Table~\ref{tab:Fisher} describes all $13$ terms that
were significant and contained at least $3$ genes. The effect of high
temperature on metabolism is profound, and it is not surprising that
the enriched GO terms include many generic stress and stress response
genes. They also include $5$ of $7$ genes with aminoacylase activity,
and $4$ of $17$ genes involved in peroxidase activity. Stress induced
formation of reactive oxygen species (ROS) can have deleterious
effects, and peroxidase activity has been seen in other organisms. In
many plants, including strawberry, high temperature adaptation led to
an increase in expression of ROS scavenging genes, including
peroxidase, and a reduction in total protein content due to protein
denaturation, and inhibited synthesis~\cite{gulen2004effect}, and
these effects are also observed in potato. The genes include Pxd
(CG3477), the peroxidase component of the
chorion~\cite{konstandi2005enzymatic}. Chorion peroxidases are
conjectured to play a role in ROS defense during egg formation in
mosquitoes~\cite{li2006major}.


One of the difficulties in genome-wide association studies is that the
number of polymorphisms are different between genes, i.e., longer
genes contain more SNPs and larger number of polymorphisms increase
the chance of showing association for larger genes.  Although the
\comale\ statistic for genes does not favor longer genes, we performed
SNP-based GO enrichment and compared enrichment versus the window
based approach. We tested associations of top 1\%-ile of the SNPs
within the $21$ candidate intervals (Table~\ref{tab:intervals}) using
\texttt{Gowinda}~\cite{kofler2012gowinda} and found 325 GO terms (see
supplemental table S4) with FDR $\le 0.001$. $9$ of the $13$
previously enriched terms (Table~\ref{tab:Fisher}), overlapped with
the $325$ \texttt{Gowinda} terms (Fisher exact $p$-val: $10^{-11}$)
suggesting consistency of SNP based and window-based analysis.
