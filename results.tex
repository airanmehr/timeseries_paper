\section{Results}
\paragraph{Modeling neutral trajectories in finite populations.} 
We tested the closeness of fit for the Markov Likelihood as a model
for neutral trajectories, compared to Brownian motion. We performed
$150$K simulations for different values of $\nu_0$
($\nu_0\in\{0.005,0.1\}$) and time $\tau$ generations $\tau\in
\{1,10,100\}$.  Fig.~\ref{fig:markov} shows that Brownian motion is
inadequate when $\nu_0$ is far from $0.5$, and when sampling is done
after many generations $\tau>1$. (sampling times are sparse). In most
experimental evolution scenarios, a site is unlikely to have frequency
close to $0.5$, and the starting frequencies are usually much
smaller. Moreover, sampling times are sparse. In typical
\emph{Drosophila} experiments for example, $<10$ time points are
samples in a span of $100$ generations from the onset of
selection~\cite{orozco2012adaptation, zhou2011experimental}.

In contrast, Fig.~\ref{fig:markov}A-F also shows that Markov
Likelihood predictions (Eq.~\ref{eq:mkvs}) are highly consistent with
empirical data for a wide range of simulation parameters. We also
tested the model under selection by conducting $100$K simulations with
selection strength $s=0.1$ on a site with initial frequency
$\nu_0=0.005$ and sampling after $\tau$ $(\tau\in\{1,10,100\})$
generations. The empirical and theoretical distributions tracked
closely (Fig.~\ref{fig:markov}G-I).

\paragraph{Detection Power.} 
To evaluate performance of each method, define power as the fraction
of true positives identified with false-positive rate $\le 0.05$
(Fig.~\ref{fig:powerROC}). Before comparing against other methods, we
first evaluated the use of HMM and Markov chain composite likelihoods
($\Hc$ and $\Mc$) in \comale\ with different percentile-cutoffs $\pi$
(Eq.~\ref{eq:pihmm}) under different sequence coverage settings. We
chose the sequence depth of each marker by identically and
independently sampling from a Poisson distribution with parameter
$\lambda\in \{30,100,\infty\}$. We computed power for ${\cal H}_\pi$
and ${\cal M}_{\pi}$, with $\pi \in \{0,0.99,100\}$. See
Fig.~\ref{fig:powerCLR}. The performance of HMM is robust with
coverage, while the Markov chain's power decays for low coverage
values. Also, the composite likelihood for all variants ($\pi=0$)
shows the highest power (Fig.~\ref{fig:powerCLR}, and
Table~\ref{tab:power}).  Therefore, we used ${\cal H} = {\cal H}_0$ as
the default statistic for all subsequent calculations.

We compared the power of $\Hc$, $\Mc$ Gaussian process
(GP)~\cite{Terhorst2015Multi}, FIT~\cite{feder2014Identifying}
statistics.  All methods other than \comale\ convert read counts to
allele frequencies and compute their test statistic. For each
experiment, (specified with values for selection coefficient $s$,
starting allele frequency $\nu_0$, coverage $\lambda$, sampling time
schedule ${\cal T}$, and number of replicates $R$), we conducted
$1000$ simulations. Half of these modeled neutral evolution and the
rest were under selection. \comale\ shows the highest power across
these range of parameters (Fig.~\ref{fig:power}). However, for perfect
coverage ($\lambda=\infty$), and high selection coefficient $s=0.1$,
GP has comparable power.




\paragraph{Running Time.}
As \comale\ does not compute full likelihoods, or explicitly model
linkage between sites, the complexity of computing likelihoods is
$\Oc(TR)$, and can be efficiently vectorized for multiple replicates
and loci. Therefore, it is expected to be faster than other approaches
like Gaussian Process (GP)~\cite{Terhorst2015Multi}. We conducted
$1000$ simulations and measured running time for \comale\ and
GP. \comale\ is $\sim 10^3\times$ faster than single locus GP (Fig.
\ref{fig:runTime}), while maintaining high power. These times have a
practical consequence. To run GP in the single locus mode on the
entire Drosophila genome of a small sample (2M variant sites), would
take 1444 CPU-hours ($\approx$ 1 CPU-month). In contrast, \comale\ ran
in 22 CPU-hours in addition to 30 mins. for precomputation of
transition matrices.  In addition, core computations of \comale\ are
matrix products, which can be efficiently vectorized. We vectorized
operations using \texttt{numba} package, which reduced run time of
over 1.5M variant sites to less an hour.


\paragraph{SFS for Detection in Natural Samples.} We did not show the
SFS based statistics in Fig.~\ref{fig:power} as they did not perform
better than random. In many experimental evolution settings, we sample
a restricted set of $F$ founder lines, where $F<<N_e$
(Fig.~\ref{fig:ee}B). This creates a severe bottleneck, confounding
SFS. The Supp. Fig.~\ref{fig:bottleneck} demonstrates the effect of
experimental evolution on different SFS statistics under neutral
evolution for 1000 simulations. A second problem with using SFS for
experimental evolution is that the sampling starts right after the
onset of experimentally induced selection, and the favored allele may
not reach high enough frequency to modify the site frequency spectrum
(Suppl. Fig.~\ref{fig:sweep}).

However, in experiments involving naturally occurring populations,
even if the span of the time-series is small, the onset of selection
might occur many generations prior to sampling. To test performance of
SFS-based statistics in natural evolution, we conducted 200 (100
neutral and 100 sweep) forward simulations for different values of
$s,\lambda$ using $N_e=10K$ and accumulating new mutations. The start
of sampling was done at a randomly picked time subsequent to the onset
of selection in two distinct scenarios. Let $t_{\nu=x}(s,N_e)$ denote
the expected time (in generations) required to reach carrier frequency
$x$ in a hard sweep and $U[a,b]$ denote discrete uniform distribution
in the interval $[a,b]$. First we considered the case when start of
sampling is chosen throughout the whole sweep. i.e., $\tau_1 \sim
U\left[1,t_{\nu=1}(s,N_e)\right]$
(Fig.~\ref{fig:powerSFS}A). Next, we considered sampling start time
  chosen nearer to fixation of the favored allele, i.e., $\tau_1 \sim
  U\left[t_{\nu=0.9}(s,N_e),t_{\nu=1}(s,N_e)\right]$
  (Fig.~\ref{fig:powerSFS}B). In both scenarios, sampling was done
  over $5$ time points within $50$ generations of $\tau_1$. We
  compared $\Hc$, GP, FIT with both static and dynamic SFS based
  statistics of SFSelect and Tajima's D. Fig.~\ref{fig:powerSFS}A
  shows that SFS based statistics are outperformed by single locus and
  CLR methods. However, when sampling is performed close to fixation,
  i.e., when the favored allele has frequency of 0.9 or higher, SFS
  based statistics perform significantly better than GP, FIT and $\Hc$
  (Fig.~\ref{fig:powerSFS}b). Moreover, dynamic SFS statistics
  outperform static SFS statistics, demonstrating that in these
  scenarios SFS based statistics should be used to detect selection.


\paragraph{Locating the Adaptive Mutation.}
The secondary task in identifying selection is to locate the position
of the adaptive allele. We simply consider the site with highest score
in the window as the locus of the favored allele. For each setting
of $\nu_0$ and $s$, we conducted 500 simulations
and computed the rank of the favored mutation in each
simulation. We plotted the cumulative distribution of the rank in
Fig.~\ref{fig:rank}. In all configuration \comale\ ranks favored allele higher 
than GP. In particular when selection is strong ($s=0.1$), \comale\ confidently 
picks the beneficial allele, i.e., when $s=0.1$, the beneficial allele is 
ranked first in 99\% of the soft sweep simulations and ranked first in 95\% of 
the simulations.
 
\paragraph{Strength of Selection.}
As the \comale\ likelihood calculation is model based, we can also
compute the model parameters $\hat{s}$ that maximized
likelihood~\eqref{eq:hmmlik}. We computed bias, $s-\hat{s}$ for each
experiment of \comale, and compared it against GP. The distribution of
the error (bias) for 100X coverage is presented in Fig.~\ref{fig:bias100} for 
different
configurations. Also 
Fig.~\ref{fig:bias30},~\ref{fig:bias300},~\ref{fig:bias3inf} depict 
distribution of estimation error for 30X, 300X and infinity coverage, 
respectively.
In general, both GP and \comale\ have biased results
for weak selection, where genetic drift dominates. However, for
stronger sweeps, e.g.$s=0.1$, \comale\ provides estimates with smaller
bias and variance. Standard deviation of the bias of $\Hc$ is 0.028 and 0.011 
in hard and soft sweep scenarios, while it is 0.036 and 0.013 for GP, 
respectively.



\ignore{
\paragraph{SFS based statistics in time series.} \VB{We need to delete, and 
move some of this earlier.}
We also considered SFS based tests including Tajima's D, Fay \& Wu's H and 
SFSelect for detecting selection. 
As these tests work only on static data, we extend them for time series data 
for both null and alternative hypotheses.
More precisely, we explicitly defined functionals $D_t$ and $H_t$ as function 
of initial carrier frequency $\nu_0$ and strength of selection $s$ (see section 
\ref{sec:sfs-ts} for details).
 To show that the proposed models of $D_t$ and $H_t$ are valid models, we 
 simulated 1000 populations and computed SFS based statistics every 10 
 generation and compared them with the proposed model.
 As shown in the Fig.~\ref{fig:sfs} proposed models are more consistent with 
 data.

 \paragraph{SFS based tests will fail when carrier frequency $\nu_t$
   is low.} \VB{We need to delete, and move some of this earlier.}
 Importantly, Fig.~\ref{fig:sfs} shows the increase in variance of
 the trajectories associated with SFS, compared to carrier
 frequency. Moreover, it is difficult to distinguish SFS trajectories
 of selection and neutral populations, when carrier frequency is
 small, e.g. first 50 generations of Fig.~\ref{fig:sfs}.  This,
 observation can be verified by examining the terms in the functional
 form of $D_t$.  As shown in the Fig.~\ref{fig:tdterms} right, in
 early generations of hard sweep where carrier frequency is low, $D_t$
 is either positive or close zero. In other words, the reduction in
 diversity become significant when carrier frequency is high enough.
 It can be shown that this argument holds for $H_t$ in early
 generations of hard sweep, where carrier frequency is not high.

}





\subsection{\data}\label{sec:dmel}
We applied \comale\ to a controlled experimental evolution
experiment~\cite{orozco2012adaptation}, where $3$ replicate samples
were chosen from a population of \emph{Drosophila melanogaster} for 59
generations under alternating 12-hour cycles of hot (28$^{\circ}$C)
and cold (18$^{\circ}$C) temperatures and sequenced. The sampling read
depths are highly heterogeneous
(Suppl. Figs.~~\ref{fig:depth},~\ref{fig:depthHetero}). Filtering low
coverage sites from data can dramatically reduce data available for
analysis. For example, by setting minimum read depth at a site to be
$30$, allowing the site to be retained only if the depth for all time
points, and all replicates exceeded $30$, the number of sites analyzed
would drop from from 1,544,374 to 10,387. Instead, \comale\ computes
$\Hc$ statistic for all sites where the sites with low coverage will
take lower scores by \comale.

We computed the \comale\ ${\cal H}$ statistic for sliding
window of 50Kbp with steps of 10Kbp over the whole genome. We observed
a spurious negative correlation between $\Hc$ and the number of SNPs
in each window (Suppl. Fig.~\ref{fig:manhattan}). To make this
precise, we computed the Pearson correlation between the number of
variants $w$ in a window, and $\Hc$ as $\rho(w,\Hc)=-0.03$
overall. However, when restricted to the candidate regions with a high
$\Hc$ value, we found $\rho(w,\Hc)=-0.27$ . This nine-fold increase in
correlation indicates that $\Hc$ statistic takes more extreme values
when SNPs are depleted. Note that the expected number of variants in a
$50$Kbp window equals $1175$ by Watterson's estimate. Therefore, we
filtered out genomic regions containing fewer than $500$ SNPs. In
repeating the test, the average PCC for all windows was computed to be
$\rho(w,\Hc)=-0.08$, and for windows with a high value of $\Hc$, it
remained close at
$\rho(w,\Hc)=-0.09$. Fig.~\ref{fig:manhattancutoffed} shows a
Manhattan plot of scores. Our results are consistent with previous
studies~\cite{orozco2012adaptation} in showing an over-representation
of significant variants on Chromosome 3R.

The 1\%-ile cut-off reveals $25$ distinct intervals 
(Table~\ref{tab:intervals})  spanning $243$ of 
the $13,965$ genes that encoded
variants. We found $36$ GO terms associated with ``Biological Process" to be 
over-represented in the $243$ genes using a Fisher exact
test for significance. Table~\ref{tab:Fisher} describes all terms that
were significant, and contained at least $3$ genes. The
effect of high temperature on metabolism is profound, and it is not
surprising that the enriched GO terms include many generic stress and
stress response genes. They also include $5$ of $7$ genes with
aminoacylase activity, and $4$ of $17$ genes involved in peroxidase
activity. Stress induced formation of reactive oxygen species (ROS)
can have deleterious effects. In many plants, including strawberry,
high temperature adaptation led to an increase in expression of ROS
scavenging genes, including peroxidase, and a reduction in total
protein content due to protein denaturation, and inhibited
synthesis~\cite{gulen2004effect}, and these effects are also observed in
potato. The genes include Pxd (CG3477), the peroxidase component of the
chorion~\cite{konstandi2005enzymatic}. Chorion peroxidases are conjectured
to play a role in ROS defense during egg formation in 
mosquitoes~\cite{li2006major}. 


One of the difficulties in genome-wide association studies is that the number 
of polymorphisms are different between genes, i.e., longer genes contain more 
SNPs and larger number of polymorphisms increase the chance of showing 
association for larger genes. 
Although \comale\ statistic for genes does not favor longer genes, we here 
perform SNP-based
GO enrichment to compare enrichment of two different approaches. We tested 
associations of top
1\%-ile of the SNPs within the 21 candidate intervals 
(Table~\ref{tab:intervals}) using Gowinda~\cite{kofler2012gowinda} and found 325
GO terms (see supplemental table S4) with FDR≤0.001. 
Also, to show that both analysis lead to
the similar results, we performed Fisher exact test on the 13 previously 
enriched (Table~\ref{tab:Fisher}) and
325 newly enriched GO terms. Given total GO terms and intersection of 9, Fisher 
exact test provides $p$-value of 10$^{-11}$.

\ignore{
\paragraph{Dominance.}
The value of the overdominance parameter can provide an insight into the kind 
of adaptation \emph{using population frequency data}. In fact, for $s>0$ we 
	have~\cite{gillespie2010population}  
	\AI{results to be added }
\begin{center}
	\begin{tabular}{l|c}
		condition & comment\\
		\hline
		$h<0$ &  underdominance\\
		$h=0$ & recessive adaptive allele\\
		$h=0.5$ & directional selection\\
		$h=1$&	dominant adaptive allele	\\
		$h>1$ &overdominance
	\end{tabular}
\end{center}}
\paragraph{Software.}
The source code and running scripts for \comale\ are available at \\
\href{https://github.com/bafnalab/comale}{https://github.com/bafnalab/comale}.