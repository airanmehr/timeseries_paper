\section{Methods}
In this section we formally present describe RNN model and a Naive method which takes $\Oc(1)$ computations as baseline performance.

\subsection{Notation}
In this paper, we denote Tensor with bold-faced capital letter, matrices with capital letter, vectors with bold-faced small letter, and scalars with small letters. Let $\bfX \in [0,1]^{R \times T \times L}$  be the Tensor of containing allele frequencies of the population where $R,L,T$ are number of experimental replicates, samples in time, segregating sites, respectively. Also, in our simplified notation $X_r=[\bfx_1 ,\cdots, \bfx_L]\in  [0,1]^ {T \times L}$  data matrix for replicate $r$,  $\bfx_l =[x_1,\cdots,x_T]$ is the observation at locus $i$ for a given replicate, and $x_t\in [0,1]$ is the allele frequency at time $t$ for a given replicate and locus. It should be noted that $T=|\Tc|$ where $\Tc=\{\tau_1,\tau_2,\cdot, \tau_T \}$ is the set of times (generations) of samples which pool sequencing data is collected. For examples $\Tc=\{2,18,25\}$ denotes that the dataset contains allele frequency of the population at 2nd, 18th and 25th generations.



\subsection{Recurrent Neural Network}
For simplicity, in this paper we consider bi-allelic single-locus Wright-Fisher (WF) model with no mutations \cite{book-mathpopgen} which assumes  discrete generation, random mating etc. Under finite population size\footnote{In infinite population size we have $x_{t+1} = f(x_t;s,h)$.} assumption allele frequency at each generation is a random variable
\beq\label{eq:trans0}
x_{t+1} = f(x_t;s,h)  + \epsilon_t
\eeq
where $h$ is the overdominance, $\epsilon_t$ is random noise due to genetic drift at generation $t$ and $f: [0,1] \mapsto [0,1]$ is the transition function:
\beq
f(x)=\frac{(1+s)x^2 + (1+hs)x(1-x)}{(1+s)x^2 + 2(1+hs)x(1-x) + x(1-x)}=x+\frac{s(h+(1-2h)x)x(1-x)}{1+sx(2h+(1-2h)x))}
\eeq
Since the transition function depend on both $s$ and $h$, we can estimate both of them at the same time by solving a 2-dimensional optimization problem. However, for the sake of simplicity of notation we aim to only estimate $s$ and set $h=0.5$ 
\beq
f(x)=x+\frac{sx(1-x)}{2+2sx}
\eeq
Given $x_t$, we can consider \ref{eq:trans0} as a nonlinear function of $s$ and finding $s$ can be formulated as a classical a 1-dimensional non-linear least-squares problem \cite{•}
\beq \label{eq:nlls0}
s^*=\underset{s}{\arg \min} \| x_{t+1} -f(x_t;s,h)\|^2
\eeq
which can be solved by iterative optimization algorithms. 
However, \eqref{eq:nlls0} does not account for dependence of $x_t,x_{t-1}, \cdots$ to $s$. To incorporate this dependence, we introduce the notation
\beq
y_t(s)= \underbrace{f(\cdots f(}_tx_0;s))
\eeq
and we can find $s$ by solving
\beq
s^*=\underset{s}{\arg \min} \sum_{t=1}^T \| y_{\tau_t}(s)- x_{t} \|^2
\eeq
On the other hand, using ordinary differential equations (ODE), \eqref{eq:trans0} can be written as \cite{multilocus-hitchhike}:

\beq
y_t(s)=\frac{x_0}{x_0 +(1-x_0)e^{-st/2}}= \frac{1}{1+ce^{-st/2}} = \sigma(-st/2)
\eeq
where $c=(1-x_0)/x_0$ is constant and $\sigma(.)$ is modified sigmoid function (due to presence of $c$). Therefore, we have
\beq
s^*=\underset{s}{\arg \min} \sum_{t=1}^T \| \sigma(s\tau_t/2)- x_{t} \|^2
\eeq
which is a standard 1-d nonlinear least squares optimization and can be solved iteratively .

\newpage


%\subsection{Gaussian Process}
%The Gaussian Process optimizes
%\beq
%\underset{\theta}{ \arg \max} \ \ \Lc(\bfX | \theta)
%\eeq
%where $\Lc$ is Gaussian distribution log-likelihood, i.e. negative weighted least-squares loss. Mean and covariance functions of the GP at any $t,l$ are functionally dependent to parameter of interest $\theta$, and computed using transition function of the WF process \cite{EandR-GP}.
%\newpage
%\cite{ilya-thesis} Backpropagation algorithm\citep{backprop}
%
%
%We define the notation $y_t=f^{(t)}(x_0)$ applying the function $f$ 
%
%
%Given $x_0$ we define $y_t$ as predicted allele frequency 
%
%\begin{figure}[h!]
%  \centering
%    \includegraphics[width=0.7\textwidth]{EandRRNN.png}
%      \caption{RNN}
%\end{figure}
%
%\subsection{Naive Method}
%
%where solving for $s$ leads to
%\beq
%s=2 t \log \left( \frac{x_t (1-x_0)}{x_0 (1-x_t)} \right)
%\eeq
