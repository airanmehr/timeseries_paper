\section{Discussion}
Experimental
evolution approaches along with time-series sequencing of the evolving
population are being applied in an increasing number of scenarios. In
the development of \comale, we show that dealing with heterogeneous data, it is 
possible to achieve significantly higher detection power as well as 
computational efficiency.


A serious constraint in
adaptive experimental evolution experiment is the
\emph{sampling-time-span}, the number of generations between the first
and last sampled generations. Given a fixed amount of time for a
study, sampling-time-span depends upon the generation time of the
organism, and can often be small relative to fixation time. For
example $30$-$50$ generations is typical for \emph{Drosophila}, with
some exceptions~\cite{zhou2011experimental}.  Therefore, unless the
selection coefficient is very strong the time series data will only
capture a ``partial sweep''.  This limitation is more critical in the
controlled experimental evolutions, where the sampling starts at the
onset of selection, and favored allele grows in frequency very slowly.
On the other hand, in sampling Natural populations, the time of onset
of selection, may not be known. In a second contribution, we extend
SFS based statistics to analyze dynamic data. We demonstrate the
strengths and weaknesses of the dynamic SFS statistics, and show that
they outperform other methods in when dynamic data is collected from
natural populations with a higher frequency of the favored allele,
i.e., the favored allele is closer to fixation.

Second,
in controlled experimental evolution experiments, populations are
evolved and inbred. This scenario in which population size
significantly drops from the large number of wild type (e.g., for
\emph{Drosophila}, ($N_e\approx10^6$) to a small number of founder
lines $F$ ($\approx 10^2$) for Experimental Evolution, resembles a
severe population bottleneck. Such a intense reduction in effective
population size have confounds both dynamic-SFS statistics and dynamic
site allele frequency based statistics\footnote{Smaller population
	size implies stronger genetic drift. In small population
	experiments, drift can be so strong that the frequency of a site
	changes rapidly so that it predicted as a site under selection.}
which leads to high false-positives.


In our simulations, we found that the power of detection can be severely 
affected by sampling regime as well as initial state of beneficial allele. In 
experimental evolution, the span of sampling is the 
. Many experimental evolution methods start sampling at the
onset of selection, and continue up to $50$ or so generations. For
small values of the selection coefficient, this may not be
sufficient. However, even if it were possible to sample over a larger
time-span, many methods, especially the ones that compute full
likelihoods cannot compute evolutionary trajectories over a large span
of generations. In contrast \comale\ precomputes the transition
matrices, and can work for any span and time of sampling.

We also show to extend SFS based methods to handle time-series
data. In initial experiments, we found that these methods do not fare
well in the traditional regimes of sampling. However, for many naturally
occurring populations, it may be advantageous to sample many
generations after the onset of selection, when the favored allele is
close to being fixed. In those scearios, SFS based methods can indeed
provide higher power.