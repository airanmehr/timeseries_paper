\section{Discussion}
Experimental
evolution approaches along with time-series sequencing of the evolving
population are being applied in an increasing number of scenarios. In
the development of \comale, we show that it is important to make
appropriate choices to achieve high power as well as computational
efficiency.

In our simulations, we found that the power of a method also depends a
lot on when the populations are sampled, and the span of
sampling. Many experimental evolution methods start sampling at the
onset of selection, and continue up to $50$ or so generations. For
small values of the selection coefficient, this may not be
sufficient. However, even if it were possible to sample over a larger
time-span, many methods, especially the ones that compute full
likelihoods cannot compute evolutionary trajectories over a large span
of generations. In contrast \comale\ precomputes the transition
matrices, and can work for any span and time of sampling.

We also show to extend SFS based methods to handle time-series
data. In initial experiments, we found that these methods do not fare
well in the traditional regimes of sampling. However, for many naturally
occurring populations, it may be advantageous to sample many
generations after the onset of selection, when the favored allele is
close to being fixed. In those scearios, SFS based methods can indeed
provide higher power.