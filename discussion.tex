\section{Discussion}

We developed a computational tool \comale, that can detect regions
under selection experimental evolution experiments of sexual
populations. Using extensive simulations, we show that \comale\
outperforms existing methods in detecting selection, locating the
favored allele, and estimating selection parameters. Importantly, we
make design choices that make \comale\ very fast in practice, allowing
us to scan all variants in genome-wide studies.


Many factors play a role in adaptation during experimental evolution
studies. The statistics used by \comale\ perform well because they
account for many of these aspects, including complete and exact
modeling of time-series data, not restricted to two-time point
time-series, correction of ascertainment bias for finite-depth
pooled-seq or finite sample variant data, and linkage within a region
using composite likelihood statistics. \comale\ uses $s,h$ as model
parameters in its likelihood calculation, and provides optimized
estimates of these parameters.

In general, while EE studies are powerful, they also pose some
challenges that are not adequately considered by other tools. One
serious constraint is the \emph{sampling-time-span}, the gap between
the first and last sampled generations, which depends upon the
generation time of the organism. It can be very small relative to the
time of fixation of the favored allele. In \emph{Drosophila} for
example, $30$-$50$ generations are typical, although there are some
notable exceptions~\cite{zhou2011experimental}.  Therefore, unless the
selection coefficient is very strong the time series data will only
capture a `partial sweep'.  This limitation is more critical in the
controlled experimental evolution, where the sampling often starts at
the onset of selection, and favored allele grows in frequency even
more slowly if the initial frequency of the favored allele is close to
zero (hard sweep). Through careful modeling, \comale\ performs better
than competing tools in these constrained scenarios.

In controlled experimental evolution experiments, populations are
evolved and inbred. As this scenario involves picking a small number
of founders, the effective population size significantly drops from
the large number of wild type (e.g., for \emph{Drosophila},
($N_e\approx10^6$) to a small number of founder lines $F$ ($\approx
10^2$) for Experimental Evolution, and the evolution includes a severe
population bottleneck. This bottleneck confounds SFS-based statistics
and makes it difficult to fit a model or test a hypothesis
(Fig.~\ref{fig:bottleneck}).  Hence we exclude SFS-based statistics
for experimental evolution of controlled populations when the initial
sampling time is close to the onset of selection. However, in
time-series sampling of Natural populations, the time of onset of
selection, may not be known. Even in this setting \comale\ outperforms
other tools including SFS based statistics, except in the case where
beneficial allele is close to fixation. For this specific scenario, we
developed time-dependent SFS based tests, and show excellent
performance compared to all other tests. Both \comale\ and SFS based
tests are implemented in the \comale\ software. In the default mode,
\comale\ computes scores for all tests, to identify regions under
selection.
	

In our simulations, we found that the power of detection can be
severely affected by sampling schedule as well as initial frequency of
the favored allele. Many experimental evolution methods start sampling
at the onset of selection, and continue up to $50$ or so
generations. For small values of the selection coefficient, this may
not be sufficient, and other experiments might be possible that span
for the entire duration of the sweep. However, even if it were
possible to sample over a larger time-span, many methods, especially
the ones that compute full likelihoods, would simply not scale to
allow computation of evolutionary trajectories over a large
time-span. In contrast \comale\ precomputes the transition matrices,
and scales linearly with number of samples, irrespective of the
time-span in which they were acquired.

Sequence coverage is a practical consideration that is often ignored
by other tools.  In \comale\ we use HMMs to explicitly model variation
in coverage. Consequently, its performance is robust with decrease in
coverage relative to other tools.

The performance of tools in hard versus soft (standing variation)
scenarios is a bit counter-intuitive. On the one hand, linkage is
strongest in hard-sweep scenarios and a composite likelihood ratio
would achieve better power. On the other hand, the growth of the
favored allele (and other hitchiking alleles) is sigmoidal, and we are
closer to the exponential growth while sampling if $\nu_0$ is
higher. We model these contrasting issues by introducing the parameter
$\pi$, the percentile cut-off for including an allele in the composite
statistic. While $\pi=0$ (including all variants) is reasonably robust
in all scenarios, it works best in a hard-sweep case. However in
soft-sweep scenarios, where $\nu_0>>0$ (close to one), the beneficial
allele is at low LD with its vicinity, and larger $\pi$ works better
by filtering out sites that are not linked with the favored
allele. The automated choice for the best value of $\pi$ will be a
subject of future research. Here we show results fixing
$\pi=0$. Contextually modifying these parameters helps \comale\
achieve better performance in locating the favored allele, and future
work will be devoted to identifying the favored allele with higher
power.


Here, we applied \comale\ to a single example of experimental
evolution--analysis of temperature sensitive adaptation in
\emph{Drosophila}. However, as EE experiments with sampling of genomic
sequence increase in popularity, computational tools for identifying
the genetic basis of adaption will play a key role in observing
evolution in action.