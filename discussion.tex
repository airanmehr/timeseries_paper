\section{Discussion}
We developed a computational tool, \comale, that can detect regions
under selection experimental evolution experiments of sexual
populations. Using extensive simulations, we show that \comale\
outperforms existing methods in detecting selection, locating the
favored allele, and estimating selection parameters. Importantly, we
make design choices that make \comale\ very fast in practice, facilitates 
``genomic scan" step in genome-wide association studies.


Many factors play a role in adaptation during experimental evolution
studies. The statistics used by \comale\ perform well because they
account for many of these aspects. First by including complete and exact 
(discrete) modeling of time-series data, \comale\ is not restricted to two-time 
point time-series and provides robust predictions for all values of initial 
frequency. Second, \comale\ adjusts for heterogeneous ascertainment bias in 
finite-depth pooled-seq data to avoid hard filtering variants.
Third, \comale\ exploits presence of high linkage within a region to compute 
composite likelihood ratio statistic. 
Fourth, \comale\ uses $s,h$ as model
parameters in its likelihood calculation, and provides optimized
estimates of these parameters, which can provide extra information such as 
fixation time, and kind of selection (e.g. balancing selection).

In our simulations, we found that the power of detection can be
severely affected by sampling schedule as well as initial frequency of
the favored allele.
In general, while EE studies are powerful, they also pose some
challenges that are not adequately considered by other tools. One
serious constraint is the \emph{sampling-time-span}, the gap between
the first and last sampled generations, which depends upon the
generation time of the organism. It can be very small relative to the
time of fixation of the favored allele. In \dmel for
example, $30$-$50$ generations are typical~\cite{kofler2013guide}, although 
there are some notable exceptions~\cite{zhou2011experimental}. 
Therefore, unless the selection coefficient is very strong the time series data 
will only capture a `partial sweep'. This limitation is more critical in the
controlled experimental evolution, where the sampling often starts at
the onset of selection, and favored allele grows in frequency even
more slowly if the initial frequency of the favored allele is close to
zero (hard sweep). Through exact (discrete-time discrete-frequency) modeling, 
\comale\ performs better
than competing tools when initial frequency is low and sampling-time-span is 
limited.


However, even if it were
possible to sample over a larger time-span, many methods, especially
the ones that compute full likelihoods, would simply not scale to
allow computation of evolutionary trajectories over a large
time-span. In contrast, \comale\ precomputes the transition matrices,
and scales linearly with number of samples, irrespective of the
time-span in which they were acquired.



Sequence coverage is a practical consideration that is often ignored
by other tools.
In \comale\ we use HMMs to explicitly model variation
in coverage. Consequently, its performance is robust with decrease in
  coverage relative to other tools.
In addition, using pool-seq data with low sequencing coverages can lead to 
missing favored allele in sequencing, especially when the initial frequency is 
low. We showed that when composite statistic computed using HMM model, \comale\ 
exhibits a competitive and  stable performance. This mainly is because of 1) 
HMM consistently models for uncertainty of sequencing and 2) composite 
likelihood 
combines scores of candidate polymorphisms in a region, and even if the 
beneficial allele is being discarded due to sequencing error, its sequenced 
hitchhikers maintain the signal of selection and used in computing the 
composite likelihood.


In controlled experimental evolution experiments, populations are
evolved and inbred. As this scenario involves picking a small number
of founders, the effective population size significantly drops from
the large number of wild type (e.g., for \dmel,
($N_e\approx10^6$) to a small number of founder lines $F$ ($\approx
10^2$) for Experimental Evolution, and the evolution includes a severe
population bottleneck. This bottleneck confounds SFS-based statistics
and makes it difficult to fit a model or test a hypothesis
(Suppl.~Fig.~\ref{fig:bottleneck}).  
Hence, statistical testing based on SFS statistic provides poor performance in  
controlled experiments where the initial sampling time is close to the onset of 
selection. 
However, when sampling is started when the frequency of the favored allele is 
high (e.g., sampling from natural populations),
SFS-based methods perform  very well. In this case, larger time since the onset 
of selection provides opportunity for new mutations to deviate SFS from 
neutrality.	


\comale\ also take into account of LD in computation of the composite 
statistic, which not only provides \comale\ with better performance but also it 
reveals interesting observation in hard and soft sweeps simulations.
First, with the same selection coefficient, frequency of the favored allele 
grows faster in soft sweep (higher initial frequency) than hard sweep, which 
makes identification of selection in soft sweeps a substantially easier task.
Second, favored allele is in lower LD with its surrounding variation in soft 
sweep than hard sweep. In the most extreme case where favored allele is not in 
LD with any variant in the region, frequency of the favored allele changes 
independently of other variants. Therefore, composition of scores does not 
improve power of detection. We found that for higher initial frequency taking 
the maximum of the scores of SNPs provides better performance, while in hard 
sweep where LD with favored allele is higher, composition of scores 
significantly improve power of detecting a region as of being under selection 
or not. Interestingly, in soft sweep case where LD to beneficial allele is 
lower, significantly better results achieved for localizing favored allele than 
hard sweeps.
 



Here, we applied \comale\ to the \data~\cite{orozco2012adaptation}. \AI{real 
data discussion?} 

Finally, we note a possible future direction to adopt \comale\ can used to 
identify adaptation in other sexual and \emph{asexual} populations. 
Given preceding observations, \comale\ can potentially provide 
even better performance of detecting selection in asexual populations, since 
the performance of \comale\ is being boosted as linkage become higher and the 
favored allele is in higher LD with its surrounding variation than sexual 
population due to lack of recombination.
In site-localization and estimating 
parameters, \comale\ expected to provide similar performance as it was applied 
to sexual population.



