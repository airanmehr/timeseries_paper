\section{Introduction}
Natural selection is a key force in evolution, and a mechanism by
which populations can adapt to external `selection'
pressure. Examples of adaptation abound in the natural
world~\cite{going2016fan}, including for example, classic examples
like lactose tolerance in Northern
Europeans~\cite{bersaglieri2004genetic}, human adaptation to high
altitudes~\cite{yi2010sequencing,simonson2010genetic}, but also drug
resistance in pests~\cite{daborn2001ddt}, HIV~\cite{Feder2016More},
cancer~\cite{gottesman2002mechanisms,zahreddine2013mechanisms},
malarial parasite~\cite{ariey2014molecular,nair2007recurrent}, and
others~\cite{spellberg2008epidemic}. In these examples, understanding
the genetic basis of adaptation can provide valuable information,
underscoring the importance of the problem.


Experimental evolution refers to the study of the evolutionary
processes of a model organism in a controlled
\cite{hegreness2006equivalence,lang2013pervasive,orozco2012adaptation,
  lang2011genetic,barrick2009genome,bollback2007clonal,oz2014strength}
or natural
\cite{maldarelli2013hiv,reid2011new,denef2012situ,winters2012development,
  daniels2013genetic,barrett2008natural,bergland2014genomic}
environment. Recent advances in whole genome sequencing have enabled
us to sequence populations at a reasonable cost, even for large
genomes. Perhaps more important for experimental evolution studies, we
can now evolve and resequence (E\&R) multiple replicates of a population to
obtain \emph{longitudinal time-series data}, in order to investigate
the dynamics of evolution at molecular level.  Although constraints
such as small sizes, limited timescales, and oversimplified
laboratory environments may limit the interpretation of E\&R results,
these studies are increasingly being used to test a wide range of
hypotheses~\cite{kawecki2012experimental} and have been shown to be
more predictive than static data analysis
\cite{boyko2008assessing,desai2008polymorphism,sawyer1992population}.
In particular, longitudinal E\&R data is being used to estimate model
parameters including population
size~\cite{williamson1999using,wang2001pseudo,pollak1983new,waples1989generalized,
  Terhorst2015Multi, jonas2016estimating}, strength of
selection~\cite{mathieson2013estimating,illingworth2011distinguishing,Terhorst2015Multi,
  bollback2008estimation,illingworth2012quantifying,malaspinas2012estimating,
  steinrucken2014novel}, allele age~\cite{malaspinas2012estimating}
recombination rate~\cite{Terhorst2015Multi}, mutation
rate~\cite{Barrick2013Genome, Terhorst2015Multi}, quantitative trait
loci~\cite{baldwin2014power} and for tests of neutrality
hypotheses~\cite{feder2014Identifying,Terhorst2015Multi,burke2010genome,bergland2014genomic}.


While many E\&R study designs are being
used~\cite{Barrick2013Genome,schlotterer2015combining}, we restrict
our attention to the adaptive evolution due to standing variation in fixed size 
populations. This regime has been considered earlier, typically with
\dmel as the model organism of choice, to identify adaptive genes in
longevity and aging ~\cite{burke2010genome,remolina2012genomic} (600
generations), courtship song~\cite{turner2011population} (100
generations), hypoxia tolerance~\cite{zhou2011experimental} (200
generations), adaptation to new laboratory
environments~\cite{orozco2012adaptation,franssen2015patterns} (59
generations), egg size~\cite{jha2015whole} (40 generations), C virus
resistance~\cite{martins2014host} (20 generations), and
dark-fly~\cite{izutsu2015dynamics} (49 generations).


The task of identifying selection signatures can be addressed at
different levels of specificity. At the coarsest level, identification
could simply refer to deciding whether some genomic region (or a gene)
is under selection or not. In the following, we refer to this task as
\emph{detection}. In contrast, the task of \emph{site-identification}
corresponds to the process of finding the favored mutation/allele at
nucleotide level. Finally, \emph{estimation of model parameters}, such
as strength of selection and dominance at the site, can provide a
comprehensive description of the selection process.

\ignore{A wide range of computational methods~\cite{vitti2013detecting} have
been developed to detect regions under positive selection. A majority
of the existing methods focus on static data analysis; analysis of a
single sample of the population at a specific time, either during the
sweep, or subsequent to fixation of the favored allele. Static
analysis is focused on reduction in genetic
diversity~\cite{tajima1989statistical,fay2000hitchhiking,ronen2013learning,garud2015recent}
shift in allele-frequencies, prevalence of long
haplotypes~\cite{sabeti2006positive,vitti2013detecting}, population
differentiation~\cite{holsinger2009genetics,burke2010genome,gunther2013robust}
in
multiple-population data and others. Many existing methods use the
Site Frequency Spectrum (SFS, see \ref{fig:sfs}) to
identify departure from neutrality. Classical examples including
Tajima's \emph{D}~\cite{tajima1989statistical}, Fay and Wu's
\emph{H}~\cite{fay2000hitchhiking}, Composite Likelihood
Ratio~\cite{nielsen2005genomic}, were all shown to be weighted linear
combinations of the SFS values~\cite{achaz2009frequency}.  While
successful, these methods are prone to both, false
negatives~\cite{messer2013population}, and also false-discoveries due
to confounding factors such as demography, including bottleneck and
population expansions, and ascertainment bias ~\cite{ptak2002evidence,
  ramos2002statistical,akey2009constructing,
  nielsen2003correcting,messer2013population}. Nevertheless, SFS based
tests continue to be used successfully, often in combination with
other tests~\cite{akey2009constructing,vitti2013detecting}. One of the
contributions of this paper is the extension of SFS based methods to
analyze time-series data, and the identification of selection regimes
where these methods perform well.}

In the effort to analyze E\&R selection experiments, many authors
chose to adapt existing tests that were originally used for static
data, pairwise comparisons (two time-points) and single replicates to
perform a null scan.  For instance, Zhu \emph{et
  al.}~\cite{zhou2011experimental} used the ratio of the estimated
population size of case and control populations to compute test
statistic for each genomic region. Burke \emph{et
  al.}~\cite{burke2010genome} applied Fisher exact test to the last
observation of data on case and control populations.  Orozco-terWengel
\emph{et al.}~\cite{orozco2012adaptation} used the
Cochran-Mantel-Haenszel (CMH) test~\cite{agresti2011categorical} to
detect SNPs whose read counts change consistently across all
replicates of two time-point data. Turner \emph{et
  al.}~\cite{turner2011population} proposed the diffStat statistic to
test whether the change in allele frequencies of two populations
deviate from the distribution of change in allele frequencies of two
drifting populations. Bergland \emph{et
  al.}~\cite{bergland2014genomic} calculated $F_{st}$ to populations
throughout time to signify their differentiation from ancestral (two
time-point data) as well as geographically different populations. Jha
\emph{et al.}~\cite{jha2015whole} computed test statistic of
generalized linear-mixed model directly from read counts.

\ignore{Early \emph{direct} methods for analyzing time-series data devoted to
estimate population size in neutral populations
~\cite{williamson1999using,anderson2000monte,beaumont2003estimation,berthier2002likelihood,wang2001pseudo},
using statistical (HMM) and population genetics (Coalescent) models.
Malaspinas
\emph{et al.}~\cite{malaspinas2012estimating} extended Bollback et
al.’s method to estimate allele age in ancient-DNA (aDNA).
}


Alternatively, \emph{direct} methods have been developed to analyze
time-series data by taking a likelihood approach, and estimating
population genetics parameters.  Bollback \emph{et
  al.}~\cite{bollback2008estimation} proposed a Hidden Markov Model
(HMM) to estimate the selection coefficient $s$ and population size by
using a diffusion approximation to the continuous Wright Fisher Markov
process.  Steinr\"{u}cken and Song~\cite{steinrucken2014novel}
proposed a general diploid selection model which takes into account of
dominance of the favored allele and approximates likelihood
analytically.  Mathieson and McVean~\cite{mathieson2013estimating}
adopted HMMs to structured populations and estimated parameters using
an Expectation Maximization (EM) procedure on discretized allele
frequency.  Feder \emph{et al.}~\cite{feder2014Identifying} modeled
increments in allele frequency with a Brownian motion process,
proposed the Frequency Increment Test (FIT). More recently, Topa
\emph{et al.}~\cite{topa2015gaussian} proposed a Gaussian Process (GP)
for modeling single-locus time-series pool-seq data. Terhorst \emph{et
  al.}~\cite{Terhorst2015Multi} extended GP to compute joint
likelihood of multiple loci under null and alternative hypotheses.
Recently, Schraiber \emph{et al.}~\cite{schraiber2016bayesian}
proposed a Bayesian framework to estimate parameters using Monte Carlo
Markov chain sampling.


While existing methods have been successfully applied to their
corresponding application, they make some assumptions which may not
hold in E\&R studies.  First, they assume that the underlying
population size is large, so it is reasonable to model dynamics of
allele frequencies using continuous state models. A number of existing
methods were originally designed to process wide time spans such as
ancient DNA studies. Finally, they assume that input data is in the
form of unbiased allele frequencies, which may not be valid for
shotgun sequencing experiments.

Here, we consider a Hidden Markov Model (HMM), similar to
Williamson~\emph{et al.}~\cite{williamson1999using} and
Bollback~\emph{et al.}'s~\cite{bollback2008estimation} but under a
``small-population-size'' regime. Specifically, we use a discrete
state (frequency) model.  We show that for small population sizes,
discrete models can compute likelihood exactly, which improves
statistical performance, especially for short time-span
experiments. Additionally, we add another level of sampling-noise to
the traditional HMM model, allowing for heterogeneous ascertainment
bias due to uneven coverage among variants. We show that for a wide
range of parameters, \comale\ provides higher power for detecting
selection, estimates model parameters consistently, and localizes
favored allele more accurately compared to the state-of-the-art
methods, while being computationally efficient.
