\begin{abstract}
  Experimental evolution (EE) studies are powerful tools for observing
  molecular evolution ``in-action'' from populations sampled in
  controlled and natural environments. The advent of next generation
  sequencing technologies has made whole-genome and whole-population
  sampling possible, even for eukaryotic organisms with large genomes,
  and allowed us to locate the genes and variants responsible for
  genetic adaptation. While many computational tests have been
  developed for detecting regions under selection, they are mainly
  designed for static (single time) data, and work best when the
  favored allele is close to fixation. Conversely, EE studies provide samples over 
  multiple time points, often at early stages of selective sweep. 
  
  While more predictive than static data analysis, a majority of the
  EE studies are constrained by the limited time span since onset of
  selection, depending upon the generation time of the organism.  This
  constraint curbs the power of adaptation studies, as the population
  can only be evolved-and-resequenced for a small number of
  generations relative to the fixation-time of the favored
  allele. Moreover, coverage in pooled sequencing experiments varies
  across replicates and time points for every variant. 
	
  In this article, we directly address these issues while developing
  tools for identifying selective sweep in pool-sequenced EE of sexual
  organisms and propose Composition of Likelihoods for
  Evolve-And-Resequence experiments (\comale). Extensive simulations
  show that \comale\ achieves higher power in detecting and localizing
  selection over a wide range of parameters. In contrast to existing
  methods, the \comale\ statistic is robust to variation of
  coverage. \comale\ also provides robust estimates of model
  parameters, including selection strength and overdominance, as
  byproduct of the statistical testing, while being orders of
  magnitude faster. Finally, we applied the \comale\ statistic to
  \datadm. We identified selection in many genes, including Heat Shock
  Proteins. The genes were enriched in ``response to heat", ``cold
  acclimation'' and ``defense response to bacterium'', and other
  relevant biological processes.
\end{abstract}