\section{Discussion}
We developed a computational tool, \comale, that can detect regions
under selection experimental evolution experiments of sexual
populations. Using extensive simulations, we show that \comale\
outperforms existing methods in detecting selection, locating the
favored allele, and estimating selection parameters. Importantly, we
make design choices that make \comale\ very fast in practice,
facilitating genome-wide studies.


Many factors play a role in adaptation during experimental evolution
studies. The statistics used by \comale\ perform well because they
account for many of these aspects. \comale\ is not restricted to
two-time points, but uses the complete time-series data. Because it
uses an exact model, \comale\ achieves robust predictions for all
values of the initial frequency. It adjusts for heterogeneous
ascertainment bias in finite-depth pooled-seq data to avoid hard
filtering variants. It exploits presence of high linkage within a
region to compute composite likelihood ratio statistic. Finally,
\comale\ uses $s,h$ as model parameters in its likelihood calculation,
and provides optimized estimates of these parameters, which can
provide extra information such as fixation time, and dominance (Suppl. 
Fig.~\ref{fig:dir-bal}).

In our simulations, we found that the power of detection can be
severely affected by the sampling schedule as well as initial
frequency of the favored allele.  In general, while EE studies are
powerful, they also pose some challenges that are not adequately
considered by other tools. One serious constraint is the sampling time
span, the gap between the first and last sampled generations, which
depends upon the generation time of the organism. It can be very small
relative to the time of fixation of the favored allele. In \dmel for
example, $30$-$50$ generations are typical~\cite{kofler2013guide},
although there are some notable
exceptions~\cite{zhou2011experimental}.  Therefore, unless the
selection coefficient is very strong, the time series data will only
capture a `partial sweep'. This limitation is more pronounced in
controlled experimental evolution, where the sampling often starts at
the onset of selection. In particular, in a hard sweep scenario, the
initial frequency of the favored allele is low, and may not reach
detectable frequency in sequencing, given the sampling time
span. Through exact (discrete-time, discrete-frequency) modeling,
\comale\ performs better than competing tools even when initial
frequency is low and sampling time span is limited.


However, even if it were possible to sample over a larger time-span,
many methods, especially the ones that compute full likelihoods, would
simply not scale to allow computation of evolutionary trajectories
over a large time-span. In contrast, \comale\ precomputes the
transition matrices, and scales linearly with number of samples,
irrespective of the time-span in which they were acquired.

Sequence coverage is a practical consideration that is often ignored
by other tools. Low sequencing coverage can lead to incorrect
frequency estimates, even for the favored allele, especially when the
initial frequency is low. \comale\ uses HMMs to explicitly model
variation in sequence coverage. Moreover, it computes the composite
likelihood from multiple linked sites, reducing the impact of coverage
on any one site, and detects selection even when the favored site is
not sampled due to low sequencing depth.


In controlled experimental evolution experiments, populations are
evolved and inbred. As this scenario involves picking a small number
of founders, the effective population size significantly drops from
the large number of wild type (e.g., for \dmel, $N_e\approx10^6$) to a
small number of founder lines ($F$ $\approx 10^2$). This creates a severe 
population
bottleneck. The bottleneck confounds SFS-based statistics and makes
it difficult to fit a model or test a hypothesis
(Suppl.~Fig.~\ref{fig:bottleneck}).  Hence, statistical testing based
on SFS statistic provides poor performance in controlled experiments
where the initial sampling time is close to the onset of selection.
However, SFS-based methods perform very well when sampling is started
long after the onset of selection (e.g., sampling from natural
populations). The larger time gap from the onset of selection provides
an opportunity for the site frequency spectrum to shift away from
neutrality.



The comparison of hard and soft sweep scenarios lead to interesting
observations. First, when LD is high in the selected region, as is
often the case in a hard sweep, composition of scores significantly
improves power of detection. When LD is low, as in soft sweep
scenarios, composition of scores does not work as well. However, the
favored allele is well established at the onset of selection, and will
grow faster compared to the hard sweep scenario under identical
selection regimes. This makes it possible to detect selection even in
soft sweep scenarios. The situation is a little different with respect
to localizing the favored allele. In soft sweep scenarios, the favored
allele is not in high LD with nearby variants, and its frequency
change is independent of them. Therefore, we obtain better
localization results in soft sweep scenarios.

There are many directions to improve the analyses presented here.  In
particular, we plan to focus our attention on other organisms with
more complex life cycles and experiments with longer
sampling-time-spans. As evolve and resequencing experiments continue
to grow, deeper insights into adaptation will go hand in hand with
improved computational analysis.

\ignore{

 Therefore, it is easier to detect selection in soft
sweep. Second, favored allele is in lower LD with its surrounding
variation in soft sweep than hard sweep. In the most extreme case
where favored allele is not in LD with any variant in the region,
frequency of the favored allele changes independently of other
variants. Therefore, composition of scores does not improve power of
detection. We found that for higher initial frequency taking the
maximum of the scores of SNPs provides better performance, while in
hard sweep where LD with favored allele is higher, composition of
scores significantly improve power of detecting a region as of being
under selection or not. Interestingly, in soft sweep case where LD to
beneficial allele is lower, significantly better results achieved for
localizing favored allele than hard sweeps.
 
Finally, we note a possible future direction to adopt \comale\ can
used to identify adaptation in other sexual and \emph{asexual}
populations.  Given preceding observations, \comale\ can potentially
provide even better performance of detecting selection in asexual
populations, since the performance of \comale\ is being boosted as
linkage become higher and the favored allele is in higher LD with its
surrounding variation than sexual population due to lack of
recombination.  In site-localization and estimating parameters,
\comale\ expected to provide similar performance as it was applied to
sexual population.
}


