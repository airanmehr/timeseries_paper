
%%%%%%%%%%%%%%%%%%%%%%%%%%%%%%%%%%%%%%%%%%%%%
%%%%%%%%%%%%%%%%%%% Supplemental Material%%%%
%%%%%%%%%%%%%%%%%%%%%%%%%%%%%%%%%%%%%%%%%%%%%
\clearpage
%\newpage
%\setcounter{page}{1}
\setcounter{figure}{0}
\setcounter{table}{0}
\setcounter{equation}{0}
\renewcommand{\thefigure}{S\arabic{figure}}
\renewcommand{\thetable}{S\arabic{table}}
\renewcommand{\theequation}{S\arabic{equation}}

%\section*{Main text figure captions}
%\section*{Supporting information captions}
\section{Appendix}
\subsection{An approximate logistic function  for allele frequency dynamics} 
\label{app:af}
Assume that a site is evolving under selection constraints $s,h\in
\Rbb$, where $s$ and $h$ denote selection strength and dominance,
respectively. Let $\nu_t$ denote the frequency of the site at time
$\tau_t\in {\cal T}$. Then, $\nusp$, the frequency at time $\tau_t+1$
can be estimated (See Eq.~\ref{eq:transition}) using:
\[\hat{\nu}_{t^+}
=\nu_t+\frac{s(h+(1-2h)\nu_t)\nu_t(1-\nu_t)}{1+s\nu_t(2h+(1-2h)\nu_t))}.
  \label{eq:transition}
\]
We can show that the dynamic of the favored allele can be modeled
via a logistic function, in the case of directional selection
($h=0.5$). Taking derivatives of Eq.~\ref{eq:transition}, we have
\beq \frac{\bfd \nu_t}{\bfd t} = \frac{s\nu_t(1-\nu_t)}{2+2s\nu_t}
\eeq To, solve the differential equation, note that for small $s$,
$2+2s\nu_t \approx 2$. Substituting,
\begin{equation}
\nu_t =\frac{1}{1+\frac{1-\nu_0}{\nu_0}e^{-st/2}} = \sigma(st/2+\eta(\nu_0)) 
\label{eq:inf-pop}
\end{equation}
where $\sigma(.)$ is the logistic function and $\eta(.)$ is logit function 
(inverse of the logistic function). 

\subsection{Dynamic of Tajima's D}\label{app:td}
In this part we derive dynamic of Tajima's $D$ statistic in \emph{hard
  sweep} as function of its value at the onset of selection, $D_0$,
selection strength and the frequency of the favored allele at the
onset of selection.  Let $D_0, \Pi_0, W_0$, be Tajima's D, Tajima's
estimate of $\theta$, and Watterson's estimate of $\theta$ at time
zero and $D_0=\Pi_0 - W_0$.  In order to compute, $D_t=\Pi_t - W_t$ we
compute $\Pi_t$ and $W_t$ separately as follows. Let $P$ be the $n
\times n$ matrix of pairwise heterozygosity if individuals, then
$\Pi=\frac{1}{n^2}\sum P_{ij}$. So, if the population consist of $\nu
n$ identical carrier haplotype (due to lack of recombination), their
pairwise hamming distance is zero and should be subtracted from the
total $\Pi_t$: \beq \Pi_t&= (1-\nu_t^2)\Pi_0 \label{eq:tdt0} \eeq

To compute $W_t$, first remember that $W_t= \frac{m_t}{S_n}$ where
$m_t$ is the number of segregating sites at time $t$ and $S_n=
\sum_i^n 1/i \approx \log(n)$. Also we have \beq
\frac{W_t}{W_0}&=\frac{\frac{m_t}{S}}{\frac{m_0}{S}} \ \ \Rightarrow
W_t=\frac{m_t}{m_0}W0 \label{eq:tdt1} \eeq Because of hard sweep and
lack of recombination assumption, the population at time $t$ consist
of $(1-\nu_t)n$ non-carrier haplotypes and $\nu_tn$ identical carrier
haplotypes. While not strictly correct, we assume that the
$(1-\nu_t)n+1$ individuals are evolving neutrally. Using this
assumption, we have \beq \frac{m_t}{m_0}&=\frac{\log\left((1-\nu_t)n
    +1 \right)\theta}{\log(n)\theta} \approx
\frac{\log\left((1-\nu_t)n\right)}{\log(n)} =
\frac{\log(1-\nu_t)+\log(n)}{\log(n)} = 1+ \frac{
  \log(1-\nu_t)}{\log(n)}. \label{eq:tdt2} \eeq Finally, by putting
Eqs.~\ref{eq:tdt0},~\ref{eq:tdt1},~\ref{eq:tdt2} together, we can
explicitly write the dynmics of $D$ statistic as \beq D_t&=
(1-\nu_t^2)\Pi_0 - (1+ \frac{ \log(1-\nu_t)}{\log(n)} ) W_0 \\&=
D_0-\log(1-\nu_t) \frac{W_0}{\log(n)} -\nu_t^2 \Pi_0\\
&\approx D_0-\log(1-\sigma(st/2+\eta(\nu_0))) \frac{W_0}{\log(n)}
-\sigma(st/2+\eta(\nu_0))^2 \Pi_0.  \eeq where $\sigma$ and $\eta$ are
logistic and logit functions.


\subsection{Dynamics of Fay and Wu's H}\label{app:h}
%\bl
In any finite population size of $n$ with $m$ segregating sites,
allele frequencies take discrete values, i.e., $x_j \in
\{\frac{1}{n},\frac{2}{n},\ldots,\frac{n-1}{n}\}, \ \forall j
\in{1,\ldots,m}$. We have the following:
\begin{equation*} 
\|\bfx\|^2= \sum_{j=1}^{m} x_j^2
= \sum_{i=1}^{n-1}\left(\frac{i}{n}\right)^2\xi_i= \frac{ (n-1)}{2n}H,
\end{equation*} 
where $\xi_i$ is the number of sites with frequency $i/n$ and $H$ is
the Fay \& Wu's estimate of $\theta$ and $\bfx \in (0,1)^m$ is the
vector of allele frequency of a region with $m$ segregating
sites. Recently, Ronen \emph{et al.}~\cite{ronen2015predicting}
devised the $1\dHAF$ statistic for identifying selection on static
data, and showed that the expected value of $1\dHAF$ statistic is
given by:
\begin{equation} 
\Ebb[1\dHAF(t)]= n\| \bfx_t\|^2\approx ng(\nu_t)
\end{equation} 
where
\beq
g(\nu_t)= \theta \nu_t \left(\frac{\nu_t+1}{2} - \frac{1}{(1-\nu_t)n+1}\right) +
\theta (1-\nu_t)\left(\frac{n+1}{2n}-\frac{1}{(1-\nu_t)n+1}\right)
\label{eq:hafscorepooled}
\eeq
The dynamics of Fay \& Wu's estimate are given by
\beq
H_t=\frac{n-1}{2} g(\nu_t)
\eeq
\subsection{Greedy computation of time-series SFS-based  
statistics}\label{app:agg}
As discussed in Section~\ref{sec:extending-sfs}, modeling dynamic of
Tajima's $D$ (and Fay\&Wu's $H$) requires knowledge of initial carrier
frequency $\nu_0$ and the value of $D$ (and $H$) statistic at the
onset of selection, which are often unknown.  As these statistics are
monotonically decreasing (or increasing for SFSelect) under no
demographic changes, we chose to greedily aggregate statistics
throughout time. For example, for Tajima's $D$, we have \beq \Dc =
\sum_{t \in \Tc} D_t \eeq where the same procedure applies to
Fay\&Wu's $H$ and SFSelect.


\subsection{Linkage Disequilibrium}
Nonrandom associations, Linkage Disequilibrium (LD), between 
polymorphisms are established in the 
substitution process, broken by recombination events 
and reinforced by selection. 
Although LD can not be measured in pooled sequencing data (phased 
haplotype data is required), it is still worthwhile 
to examine the behavior of LD as a result of the interaction between 
recombination and natural selection. In this part we theoretically overview 
expected LD in short EEs.

Let $\rho_0$ be the LD at time zero between the favored allele and a 
segregating site $l$ base-pairs away, then under natural selection we have
\beq
\rho_t= \alpha_t\beta_t \rho_0=e^{-rtl} \left(\frac{H_t}{H_0}\right)  
\rho_0\label{eq:ldt}
\eeq
where $H_t=2\nu_t(1-\nu_t)$ is the heterozygousity at the selected site, $r$ is 
the recombination rate/bp/gen. The 
\emph{decay factor}, $\alpha_t=e^{-rtl}$,
and \emph{growth factor}, $\beta_t$ (see Eq. 30-31 in 
\cite{Stephan2006The}), are result of recombination and 
selection, respectively. Fig.~\ref{fig:ld3d} presents the expected theoretical 
value of LD when $\rho_0=0.5$ between favored allele (site at position 500K) 
and the rest of 
genome, and $\nu_0=0.1$. For neutral evolution (top), LD decays exponentially 
through space and time, while in natural selection (bottom), LD increases and 
then decreases. Interestingly, LD increases to its maximum value, 1, for the 
nearby region (the plateau in the Fig.~\ref{fig:ld3d} bottom) of the favored 
allele.

In principle, LD increases after the onset of selection, until $\log(\alpha_t) 
+\log(\beta_t) 
>0$, see Eq.~\ref{eq:ldt}. 
Specifically, log of decay term is linear and, using 
Eq.~\ref{eq:inf-pop}, we write growth 
factor in term of initial frequency $\nu_0$ and selection strength $s$. 
Fig.~\ref{fig:ldf} depicts interaction of decay and growth factors for weak and 
strong selection and soft and hard sweeps. In all the case, LD of the 
favored allele with a segregating site 50Kbp away, increases in the first 50 
generations, which give rise to increasing number of \emph{hitchhikers}. 

Increase of LD in a large (100Kbp) region is particularly advantageous to the 
task of identifying the region under selection, if the composite statistics is 
used. As a result, $\Hc$ statistic outperforms existing (single-loci) tools in 
identifying selection. In contrast, augmentation of LD, increases the 
number 
of candidates for 
the favored allele, which makes is difficult to localize the favored 
allele.

\clearpage
\newpage
