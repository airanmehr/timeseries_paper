\listfiles
\documentclass[twoside,11pt]{article}
\input{macros-arya}
\usepackage{arya}
\date
\begin{document}
\title{Genomic Time-Series Modeling Using Recurrent Neural Networks}
\author{\name Arya Iranmehr  \email airanmehr@ucsd.edu \\ 
\name Vineet Bafna \email vbafna@cs.ucsd.edu \\
\name Ali Akbari \email alakbari@eng.ucsd.edu }

\maketitle
\begin{abstract}
The advent of Next Generation Sequencing (NGS) has made it possible to study genomic data throughout time. This modern paradigm, Evolve-and-Resequence (E\&R), enables us to make more accurate and robust inferences, i.e. estimate model parameters, using multiple observations along generations. In this paper, we consider the recently repopularized Recurrent Neural Networks (RNN) to model the genomic time series of E\&R. In fact, RNN is used as a generative model which for a initial estate and a choice of model parameter generates a sequence. Parameter estimation procedure involves a (non-convex) optimization of least square loss between observed sequence data and RNN-generated sequence with respect to model parameter. Backpropagation-in-time, is effectively used to compute gradients of objective function, and stochastic gradient descent with momentum algorithm is used for  optimization. Experimental study on simulated data shows RNN provides significantly more accurate and robust estimates in shorter times.
\end{abstract}
\section{Introduction}
Until very recently, biological data analysis has been considered processing a snapshot of data. However, the emergence of NGS and related technologies has made it possible to not only create larger datasets but also to measure multiple observations of the same quantity in the course of time. In many cases, such as population genetics, it is of the great interest to model the evolutionary process and make inferences, predictions and retrospective studies. Indeed, a random process is better explained by time series data than a single observation.

In addition to inexpensive data availability, over last two decades, a large amount of efforts is dedicated to High-Performance Computing (HPC), which re-popularized and re-branded computationally intensive algorithms such as Neural Networks. The first properly proposed neural network model to exploit full potential of multi layer neural networks published by \cite{deep-DR, deep-belief} and its spectacular performance on image processing problems immediately spawned the field of Deep Neural Networks (DNN), aka Deep Learning. Shortly after, DNNs went beyond the tasks that they are initially indented to accomplish \cite{deep-imagenet} and had breakthroughs in  time-series DNN models, aka RNNs, such as generative models \cite{deep-generative}, speech processing \cite{deep-speech} etc. 

In this paper we aim to use the tools and machinery that has been developed for RNNs, to model times-series biological data. In particular, we consider the population genetics problem of finding loci (locus) under selection given observation of allele frequency of a population in different generations of a Wright-Fisher model. This problem has been previously treated by using Gaussian Processes \cite{EnadR-GP}, spectral methods \cite{EandR-spectral}




\cite{multilocus-gp}
\cite{multilocus-hitchhike}
\cite{msms}
\cite{simupop}
\cite{ilya-thesis}
\cite{theano}
\citep{backprop}
\section{Discussion}
Experimental
evolution approaches along with time-series sequencing of the evolving
population are being applied in an increasing number of scenarios. In
the development of \comale, we show that it is important to make
appropriate choices to achieve high power as well as computational
efficiency.

In our simulations, we found that the power of a method also depends a
lot on when the populations are sampled, and the span of
sampling. Many experimental evolution methods start sampling at the
onset of selection, and continue up to $50$ or so generations. For
small values of the selection coefficient, this may not be
sufficient. However, even if it were possible to sample over a larger
time-span, many methods, especially the ones that compute full
likelihoods cannot compute evolutionary trajectories over a large span
of generations. In contrast \comale\ precomputes the transition
matrices, and can work for any span and time of sampling.

We also show to extend SFS based methods to handle time-series
data. In initial experiments, we found that these methods do not fare
well in the traditional regimes of sampling. However, for many naturally
occurring populations, it may be advantageous to sample many
generations after the onset of selection, when the favored allele is
close to being fixed. In those scearios, SFS based methods can indeed
provide higher power.
\section{Experiments}
In this part we compare RNLLS with GP and Naive method. 
\subsection{Data}
\subsubsection{Synthetic}
Synthetic datasets are created as follows. Firs using msms prgram, we created a population for $F$ founding haplotypes. Then a population of n homozygote diploid individuals are randomly created as initial population for each simulation. Given initial population, we used simuPop to perform forward simulation by randomly choosing the site under selection. Allele frequency of the populations at generations 10,20,30,40,50 are recorded, i.e. $\Tc=\{10,20,30,40,50\}$
\subsubsection{Real}

\input{conclusion}

\bibliographystyle{plain}
\bibliography{/home/arya/Documents/library}

\end{document}